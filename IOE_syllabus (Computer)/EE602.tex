\begin{center}
    \textbf{\huge{\uppercase{Control System}}}
    \\
    \vspace{.5cm}
    \textbf{\large{EE 602}}
\end{center}

\noindent\textbf{Lecture\ \ \ : 3} \hfill \textbf{Year : III} \\
\textbf{Tutorial \ : 1} \hfill \textbf{Part : I } \\
\textbf{Practical : 3/2}  \\

\par
\noindent 
\section*{Course Objective:}
To present the basic concepts on analysis and design of control system and to apply these concepts to typical physical processes.

\mylist{
    \textbf{Control System Background \hfill (2 hours)}
    \mylist{
        History of control system and its importance;
        Control system: Characteristics and Basic features;
        Types of control system and their comparison
    };
    \textbf{Component Modeling \hfill (6 hours)}
    \mylist{
        Differential equation and transfer function notations;
        Modeling of Mechanical Components: Mass, spring and damper;
        Modeling of Electrical components: Inductance, Capacitance, Resistance, DC and AC motor, Transducers and operational amplifiers;
        Electric circuit analogies (force-voltage analogy and force-current analogy);
        Linearized approximations of non-linear characteristics
    };
    \textbf{System Transfer Function and Responses \hfill (6 hours)}
    \mylist{
        Combinations of components to physical systems;
        Block diagram algebra and system reduction;
        Signal flow graphs;
        Time response analysis:
        \mylist{
            Types of test signals (Impulse, step, ramp, parabolic);
            Time response analysis of first order system;
            Time response analysis of second order system;
            Transient response characteristics
        };
        Effect of feedback on steady state gain, bandwidth, error magnitude and system dynamics
    };
    \textbf{Stability \hfill (4 hours)}
    \mylist{
        Introduction of stability and causes of instability;
        Characteristic equation, root location and stability;
        Setting loop gain using Routh-Hurwitz criterion;
        R-H stability criterion;
        Relative stability from complex plane axis shifting
    };
    \textbf{Root Locus Technique \hfill (7 hours)}
    \mylist{
        Introduction of root locus;
        Relationship between root loci and time response of systems;
        Rules of manual calculations and construction of root locus;
        Analysis and design using root locus concept;
        Stability analysis using R-H criteria
    };
    \textbf{Frequency Response Techniques \hfill (6 hours)}
    \mylist{
        Frequency domain characterization of the system;
        Relationship between real and complex frequency response;
        Bode Plots: Magnitude and phase;
        Effects of gain and time constant on Bode diagram;
        Stability from Bode diagram (gain margin and phase margin);
        Polar plot and Nyquist plot;
        Stability analysis from Polar and Nyquist plot
    };
    \textbf{Performance Specifications and Compensation Design  \hfill (10 hours)}
    \mylist{
        Time domain specification
        \mylist{
            Rise time, Peak time, Delay time, settling time and maximum overshoot;
            Static error co-efficient
        };
        Frequency domain specification
        \mylist{
            Gain margin and phase margin
        };
        Application of Root locus and frequency response on control system design;
        Lead, Lag cascade compensation design by Root locus method;
        Lead, Lag cascade compensation design by Bode plot method;
        PID controllers
    };
    \textbf{State Space Analysis \hfill (4 hours)}
    \mylist{
        Definition of state-space;
        State space representation of electrical and mechanical system;
        Conversion from state space to a transfer function;
        Conversion from transfer function to state space;
        State-transition matrix.
    }
}


\section*{Practical:}
\mylist{
    To study open loop and closed mode for DC motor and familiarization with different components in DC motor control module;
    To determine gain and transfer function of different control system components;
    To study effects of feedback on gain and time constant for closed loop speed control system and position control system;
    To determine frequency response of first order and second order system and to get transfer function;
    Simulation of closed loop speed control system and position control system and verification
}

\section*{References:}
\mylist{
    Ogata, K., ``Modern Control Engineering", Prentice Hall.;
    Gopal, M., ``Control Systems: Principles and Design", Tata McGraw-Hill;
    Kuo, B.C., ``Automatic Control System", Prentice Hall;
    Nagrath \& Gopal, ``Modern Control Engineering", New Ages International
}