\begin{center}
    \textbf{\huge{\uppercase{Engineering Drawing II}}}
    \\
    \vspace{.5cm}
    \textbf{\large{ME 451}}
\end{center}

\noindent\textbf{Lecture\ \ \ : 1} \hfill \textbf{Year : I } \\
\textbf{Tutorial \ : 0} \hfill \textbf{Part : II } \\
\textbf{Practical : 3}  \\

\par
\noindent 
\section*{Course Objective:}
To make familiar with the conventional practices of sectional views. To develop basic concept and skill of pictorial drawing and working drawings. Also to make familiar with standard symbols and different engineering fields.

\begin{enumerate}
    \item \textbf{ Conventional Practices for Orthographic and Sectional Views \hfill (12 hours)}
    \begin{enumerate}
        \item Conventional practices in Orthographic views: Half Views and partial views, treatment of unimportant intersections, aligned views, treatment for radially arranged features, representation of fillets and rounds.
        
        \item Conventional practices in sectional views: Conventions for Ribs, Webs and Spokes in Sectional View, Broken Section, Removed Section, Revolved Section, Offset Section, Phantom Section and Auxiliary Sectional Views
        
        \item Simplified Representations of Standard Machine Elements
        
    \end{enumerate}
    
    \item \textbf{Pictorial Drawings \hfill (20 hours)}
    \begin{enumerate}
        \item Classifications: Advantages and Disadvantages
        \item Axonometric Projection: Isometric Projection and Isometric Drawing
        \begin{enumerate}
            \item Procedure for making an isometric drawing
            \item Isometric and Non-isometric lines; Isometric and Non-isometric surfaces
            \item Angle in Isometric Drawing
            \item Circles and Circular Arcs in Isometric Drawing
            \item Irregular Curves in Isometric Drawing
            \item Isometric sectional Views
        \end{enumerate}
        \item Oblique Projection and Oblique Drawing
        \begin{enumerate}
            \item Procedure for making an Oblique Drawing
            \item Rules for Placing Objects in Oblique drawing
            \item Angles, Circles and Circular Arcs in Oblique drawing
        \end{enumerate}
        \item Perspective Projection
        \begin{enumerate}
            \item Terms used in Perspective Projection
            \item Parallel and Angular Perspective
            \item Selection of Station Point
        \end{enumerate}
    \end{enumerate}
    
    \item \textbf{Familiarization with Different Components and Conventions \hfill (8 hours)}
    \begin{enumerate}
        \item Limit Dimensioning and Machining Symbols
        \begin{enumerate}
            \item Limit, fit and tolerances
            \item Machining Symbols and Surface finish
        \end{enumerate}
        \item Threads, Bolts and Nuts
        \begin{enumerate}
            \item Thread Terms and Nomenclature, forms of screw threads
            \item Detailed and simplified representation of internal and external threads
            \item Thread Dimensioning
            \item Standard Bolts and Nuts: Hexagonal Head and Square Head
            \item Conventional Symbols for Bolts and Nuts
        \end{enumerate}
        
        \item Welding And Riveting
        \begin{enumerate}
            \item Types of Welded Joints and types of welds, welding symbols
            \item Forms and proportions for Rivet Heads, Rivet Symbols, Types of Riveted Joints: Lap Joint, Butt Joint
        \end{enumerate}
        
        \item Familiarization with Graphical Symbols and Conventions in Different Engineering Fields 
        \begin{enumerate}
            \item Standard Symbols for Civil, Structural and Agricultural Components
            \item Standard Symbols for Electrical, Mechanical and Industrial Components
            \item Standard Symbols for Electronics, Communication and Computer Components
            \item Topographical Symbols
        \end{enumerate}
        
        \item Standard Piping Symbols and Piping Drawing
    \end{enumerate}
    
    \item \textbf{ Detail and Assembly Drawings \hfill (20 hours) }
    \begin{enumerate}
        \item Introduction to Working Drawing
        \item Components of Working Drawing: Drawing Layout, Bill of Materials, Drawing Numbers
        \item Detail Drawing
        \item Assembly Drawing
        \item Practices of Detail and Assembly Drawing: V-block Clamp, Centering Cone, Couplings, Bearings, Antivibration Mounts, Stuffing Boxes, Screw Jacks, etc.
    \end{enumerate}
\end{enumerate}


\section*{Practical:}
\begin{enumerate}
    \item Conventional Practices for Orthographic and Sectional Views (Full and Half Section)
    \item Conventional Practices for Orthographic and Sectional Views (Other Type Sections)
    \item Isometric Drawing
    \item Isometric Drawing (Consisting of Curved Surfaces and Sections)
    \item Oblique Drawing
    \item Perspective Projection
    \item Familiarization and Graphical Symbol (Limit, Fit, Tolerances and Surface Roughness Symbols)
    \item Familiarization with Graphical Symbols (Symbols for Different Engineering Fields)
    \item Detail Drawing
    \item Assembly Drawing I
    \item Assembly Drawing II
    \item Building Drawing
\end{enumerate}




\section*{References:}
\begin{enumerate}
   
    \item W. J. Luzadder, ``Fundamentals of Engineering Drawing", Prentice Hall.
    \item T. E. French, C. J. Vierck, and R. J. Foster, ``Engineering Drawing and Graphic Technology", Mc Graw Hill Publishing Co.
    \item A. Mitchell, H.C. Spencer and J. T. Dygdone, ``Technical Drawing", F. E. Giescke, Macmillan Publishing Co.
    \item N. D. Bhatt, ``Elementary Engineering Drawing", Charotar Publishing House, India.
    \item P. S. Gill, ``A Text Book of Engineering Drawing", S.K. Kataria and Sons, India.
    \item R. K. Dhawan, ``A Text Book of Engineering Drawing", S. Chand and Company Limited, India.
\end{enumerate}