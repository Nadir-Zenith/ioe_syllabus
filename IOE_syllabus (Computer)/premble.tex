\section{\uppercase{Introduction}}
The Institute of Engineering(IOE) is offering this course with the objective of producing high level technical manpower capable of undertaking works in the Electronics and Communication Engineering field. The details of the course are as follows:

\subsection{Title of the Course}
    Bachelor of Engineering in \textbf{Electronics \& Communication Engineering.}
    
\subsection{Duration of the Course}
    The total duration of the course is 4 years. Each year consists of two parts, \textbf{I} and \textbf{II}, each part having duration of 90 working days(15 weeks).
    

\section{\uppercase{Course structure}}
The course is divided into 8 parts. The first year courses include fundamental common subjects. The second and third year generally including specific courses of the related discipline. The final year include professional and application type courses. 
\par
The course structure attached in the later section of this book provides information about lecture, tutorial and practical hours per week, full marks and pass marks for internal assessment and final examination, and the duration of final examination of each subject.

\section{\uppercase{Course Code}}
Each subject is specified by a unique code consisting of two letters followed by three digit number for core courses and five digit numbers for elective courses. The first two letters denote the department which offers the subject (SH: Science and Humanities, AE: Agricultural Engineering, AR: Architecture, CE: Civil Engineering, CT: Computer Engineering, EE: Electrical Engineering, EX: Electronics and Communication Engineering, GE: Geomatics Engineering, IE: Industrial Engineering, ME: Mechanical Engineering). The first digit of the number denotes the year on which the subject is offered (4 for first year, 5 for second year, 6 for third year, and 7 for fourth year respectively for Bachelor's level course). The remaining two digits 01 and 49 are used for the core subjects offered in odd parts and 51 to 99 are used for the core subjects offered in even parts. Two extra digits from 01 to 99 are used for the elective courses.
\par
\vspace{1cm}
\noindent
\textbf{Core Courses:}
\begin{table}[h]
    
    \begin{tabular}{|c|c|}
    \hline
        \textbf{AB} & \textbf{DEF}\\
    \hline
    \end{tabular}
\end{table}
\par
\textbf{AB:} Offering Department (SH, AE, AR, CE, EE, EX, GE or ME)\par
\textbf{D:} Year (4 for first year, 5 for second year, and so on). \par
\textbf{EF:} 01-49 for courses offered in odd parts and 51 to 99 for courses offered in even parts

\par
\vspace{1cm}
\noindent
\textbf{Elective Courses:}
\begin{table}[h]
    
    \begin{tabular}{|c|c|}
    \hline
        \textbf{AB} & \textbf{DEFGH}\\
    \hline
    \end{tabular}
\end{table}
\par
\textbf{GH:} 01 to 99 specific numbers to each elective course \par
\vspace{1cm}
For example, ME 751 is the code for the core course "Finite Element Method" which is offered in fourth year second part by Department of Mechanical Engineering.


\section{\uppercase{Instruction Methods}}
The method of teaching is lectured augmented by tutorials and/or practical whichever is relevant. Tutorials are used to enlarge and develop the topic and concepts stated in the lecture. Practical classes in he form of laboratory works and design/drawing practices are used to verify the concepts and to develop necessary basic skills. Each course is specified with certain lecture, tutorial and practical hour(s) per week. The hours specified as 3/2 in practical means 3 laboratory hours in each two weeks. 
\par
The use of multimedia and interactive mode (presentations) is encouraged for conducting fourth year courses.

\section{\uppercase{Internal Assessment and Final Examination}}
The students' achievement in each subject is evaluated by internal assessment and final examination.

\subsection{Internal Assessment}
20\% of the total marks is allocated for internal assessment for theory part of all subjects. Internal assessment mark should include class performance, timely submissions and correctness of assignments, class tests, quizzes, etc.
\par
Evaluation of practical part of most of the subjects is done through continuous assessment. It includes lab performance, report submission, presentation, viva, etc. However, for few courses final examinations are also conducted.
\par
70\%  attendance is mandatory to qualify for the final examination.

\subsection{Final Examination}
Final examinations of 3 hours for theoretical subjects with full marks of 80 and 1.5 hours for theoretical subjects with full mark of 40 are conducted as per academic calendar of IOE.

\subsection{Pass Marks}
Any student must obtain 40\% in both internal assessment and final examination of each subject to pass in the subject. Only students who have passed the internal assessment of a particular subject are allowed to appear in the final examination of the subject.


\subsection{\uppercase{Evaluation System}}
Students who have passed all the components of all subjects in all parts are considered to have successfully completed the course. The overall achievement of each student is measured by a final aggregate percentage which is obtained by providing a weight to percentages scored by the students in each parts as prescribed below:

\par
\noindent

\begin{table}[h]
    
    \begin{tabular}{l c}
    
        First year (both I and II parts) & 20\% \\
        Second year (both I and II parts) & 20\% \\
        Third year (both I and II parts) & 30\% \\
        Fourth year (both I and II parts) & 30\% \\
    \end{tabular}
\end{table}
\par
\noindent
Depending upon he final weighted aggregate percentage scored by a student, a division is awarded as follows:

\begin{table}[h]
    
    \begin{tabular}{l l}
    
        80\% and above & Distinction \\
        65\% or above and below 80\% & First \\
        50\% or above and below 65\% & Second \\
        40\% or above and below 50\% & Pass \\
    \end{tabular}
\end{table}