\begin{center}
    \textbf{\huge{\uppercase{Object Oriented Programming}}}
    \\
    \vspace{.5cm}
    \textbf{\large{CT 501}}
\end{center}

\noindent\textbf{Lecture\ \ \ : 3} \hfill \textbf{Year : II } \\
\textbf{Tutorial \ : 0} \hfill \textbf{Part : I } \\
\textbf{Practical : 3}  \\

\par
\noindent 
\section*{Course Objective:}
To familiarize students with the C++ programming language and use the language to develop object oriented programs.

\mylist{
    \textbf{Introduction of Object Oriented Programming \hfill (3 hours)}
    \mylist{
        Issues with Procedure Oriented Programming;
        Basic of Object Oriented Programming (OOP);
        Procedure Oriented Versus Object Oriented Programming;
        Concept of Object Oriented Programming
        \mylist{
                Object;
                Class;
                Abstraction;
                Encapsulation;
                Inheritance;
                Polymorphism
        };
        Example of some Object Oriented Languages;
        Advantages and Disadvantages of OOP
    };
    \textbf{Introduction to C++ \hfill (2 hours)}
    \mylist{
            The Need of C++;
            Features of C++;
            C++ versus C;
            History of C++
    };
    \textbf{C++ Language Constructs \hfill (6 hours)}
    \mylist{
            C++ Program Structure;
            Character Set and Tokens
            \mylist{
                Keywords;
                Identifiers;
                Literals;
                Operators and Punctuators
                };
            Variable Declaration and Expression;
            Statements;
            Data Type;
            Type Conversion and Promotion Rules;
            Preprocessor Directives;
            Namespace;
            User Defined Constant const;
            Input/Output Streams and Manipulators;
            Dynamic Memory Allocation with new and delete;
            Condition and Looping;
            Functions 
            \mylist{
                Function Syntax;
                Function Overloading;
                Inline Functions;
                Default Argument;
                Pass by Reference;
                Return by Reference
            };
            Array, Pointer and String;
            Structure, Union and Enumeration
        };
    \textbf{Objects and Classes \hfill (6 hours)}
    \mylist{
        C++ Classes;
        Access Specifiers;
        Objects and the Member Access;
        Defining Member Function;
        Constructor
        \mylist{
            Default Constructor;
            Parameterized Constructor;
            Copy Constructor
        };
        Destructors;
        Object as Function arguments and return type;
        Array of Objects;
        Pointer to Object and Member Access;
        Dynamic Memory Allocation for Objects and Object Array;
        This pointer;
        static data member and static function;
        constant member functions and constant objects;
        Friend function and Friend Classes
    };
    \textbf{Operator Overloading \hfill (5 hours)}
    \mylist{
            Overloadable Operators;
            Syntax of Operator Overloading;
            Rules of Operator Overloading;
            Unary Operator Overloading;
            Binary Operator Overloading;
            Operator Overloading with Member and Non member functions;
            Data Conversion: Basic -- User Defined and Use Defined -- User Defined;
            Explicit Constructors
        };
    \textbf{Inheritance \hfill (5 hours)}
    \mylist{
        Base and Derived Class;
        protected Access Specifier;
        Derived Class Declaration;
        Member function overriding;
        Forms of inheritance: Single, multiple, multilevel, hierarchical, hybrid, multipath;
        Multipath inheritance and virtual base class;
        Constructor Invocation in Single and Multiple inheritances;
        Destructo in single and multiple inheritances
    };
    \textbf{Polymorphism and Dynamic Binding \hfill (4 hours)}
    \mylist{
        Need of Virtual Function;
        Pointer to Derived Class;
        Definition of Virtual Functions;
        Array of Pointers to Base Class;
        Pure Virtual functions and Abstract Class;
        Virtual Destructor;
        reinterpret\_cast Operator;
        Run-time Type Information
        \mylist{
            dynamic\_cast Operator;
            typeid Operator
        }
    };
    \textbf{Stream Computation for Console and File Input/Output \hfill (5 hours)}
    \mylist{
        Stream Class Hierarchy for Console Input/Output;
        Testing Stream Errors;
        Unformatted Input/Output;
        Formatted Input/Output with ios Member functions and flags;
        Formatting with Manipulators;
        Stream Operator Overloading;
        File Input/Output with Streams;
        File Sream Class Hierarchy;
        Opening and Closing files;
        Read/Write from file;
        File Access Pointers and their Manipulators;
        Sequential and Random Access to file;
        Testing Errors during file operations
    };
    \textbf{Templates \hfill (5 hours)}
    \mylist{
        Function Template;
        Overloading Function Template
        \mylist{
            Overloading with functions;
            Overloading with other Template
        };
        Class Template
        \mylist{
            Function Definition of Class Template;
            Non-Template Type Arguments;
            Default Arguments with Class Template
        };
        Derived Class Template;
        Introduction to Standard Template Library
        \mylist{
            Containers;
            Algorithms;
            Iterators
        }
    };
    \textbf{Exception Handling \hfill (4 hours)}
    \mylist{
        Error Handling;
        Exception Handling Constructs (try, catch, throw);
        Advantage over Conventional Error Handling;
        Multiple Exception Handling;
        Rethrowing Exception;
        Catching all exceptions;
        Exceptions with arguments;
        Exceptions specification for function;
        Handling Uncaught and Unexpected exceptions
    }
}

\section*{Practical:}
There will be about 12 lab exercises covering the course. At the end of the course, students must complete a programming project on OOP with C++.


\section*{References:}
\mylist{
    Robert Lafore, ``Object Oriented Programming in C++", Sams Publication;
    Daya Sagar Baral and Diwakar Baral, ``The Secrets of Object Oriented Programming in C++", Bhundipuran Prakasan;
    Harvey M. Deital and Paul J. Deital, ``C++ How to Program", Pearson Education Inc.;
    D.S. Malik, ``C++ Programming", Thomson Course Technology;
    Herbert Schildt, ``C++: The Complete Reference", Tata McGraw Hill
}