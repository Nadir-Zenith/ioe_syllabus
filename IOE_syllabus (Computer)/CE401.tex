\begin{center}
    \textbf{\huge{\uppercase{Applied Mechanics}}}
    \\
    \vspace{.5cm}
    \textbf{\large{CE 401}}
\end{center}

\noindent\textbf{Lecture\ \ \ : 3} \hfill \textbf{Year : I } \\
\textbf{Tutorial \ : 2} \hfill \textbf{Part : I } \\
\textbf{Practical : 0}  \\

\par
\noindent 
\section*{Course Objective:}
To provide concept and knowledge of engineering mechanics and help understand structural engineering stress analysis principles in later courses or to use basics of mechanics in their branch of engineering. Emphasis has been given to Statics.

\begin{enumerate}
    \item \textbf{Introduction \hfill (2 hours)}
    \begin{enumerate}
        \item Definitions and scope of Applied Mechanics
        \item Concept of Rigid and Deformed Bodies
        \item Fundamental concepts and principles of mechanics: Newtonian Mechanics
    \end{enumerate}
    
    \item \textbf{Basic Concept in Statics and Static Equilibrium \hfill (4 hours)}
    \begin{enumerate}
        \item Concept of Particles and Free body diagram
        \item Physical meaning of Equilibrium and its essence in structural application
        \item Equation of Equilibrium in Two Dimension
    \end{enumerate}
    
    \item \textbf{Forces Acting on Particle and Rigid Body \hfill (6 hours)}
    \begin{enumerate}
        \item Different types of Forces: Point, Surface Traction and Body Forces - Translational Force and Rotational Force: Relevant Examples
        \item Resolution and Composition of Forces: Relevant Examples
        \item Principle of Transmissibility and Equivalent Forces : Relevant Examples
        \item Moments and couples : Relevant Examples
        \item Resolution of a Force into Forces and a Couple: Relevant Examples
        \item Resultant of Force and Moment for a System of Force: Relevant Examples
    \end{enumerate}
    
    \item \textbf{Center of Gravity, Centroid and Moment of Inertia \hfill (6 hours)}
    \begin{enumerate}
        \item Concepts and Calculation of Center of Gravity and Centroid: Examples
        \item Calculation of Second Moment of Area/Moment of Inertia and Radius of Gyration: And Relevant usages
        \item Use of Parallel axis Theorem: Relevant Examples
    \end{enumerate}
    
    \item \textbf{Friction \hfill (2 hours)}
    \begin{enumerate}
        \item Laws of Friction, Static and Dynamic Coefficient of Friction, Angle of friction: Engineering Examples of usage of friction
        \item Calculations involving friction in structures: Example as High Tension Friction Grip bolts and its free body diagram
    \end{enumerate}
    
    \item \textbf{Analysis of Beams and Frames \hfill (9 hours)}
    \begin{enumerate}
        \item Introduction to Structures: Discrete and Continuum
        \item Concept of Load Estimating and Support Idealizations: Examples and Standard symbols
        \item Use of beams/frames in engineering: Concept of rigid joints/distribute loads in beams/frames.
        \item Concept of Statically/Kinematically Determinate and Indeterminate Beams and Frames: Relevant Examples
        \item Calculation of Axial Forces, Shear Force and Bending Moment for Determinate Beams and Frames
        \item Axial Force, Shear Force and Bending Moment Diagrams and Examples for drawing it.
    \end{enumerate}
    
    \item \textbf{Analysis of Plane Trusses \hfill (4 hours)}
    \begin{enumerate}
        \item Use of trusses in engineering: Concept of pin joints/joint loads in trusses.
        \item Calculation of Member Forces of Truss by method of joints: Simple Examples
        \item Calculation of Member Forces of Truss by method of sections: Simple Examples
    \end{enumerate}
    
    \item \textbf{Kinematics and Particles and Rigid Body \hfill (7 hours)}
    \begin{enumerate}
        \item Rectilinear Kinematics: Continuous Motion
        \item Position, Velocity and Acceleration of a Particle and Rigid Body
        \item Determination of Motion and Particle and Rigid Body
        \item Uniform Rectilinear Motion of Particle
        \item Uniformly Accelerated Rectilinear Motion of Particles
        \item Curvilinear Motion: Rectangular Components with Examples of Particles
    \end{enumerate}
    
    \item \textbf{Kinetics of Particles and Rigid Body: Force and Acceleration \hfill (5 hours)}
    \begin{enumerate}
        \item Newton's Second Law of Motion and momentum
        \item Equation of Motion and Dynamic Equilibrium: Relevant Examples
        \item Angular Momentum and Rate of Change
        \item Equation of Motion-Rectilinear and Curvilinear
        \item Rectangular: Tangential and Normal Components and Polar Coordinates: Radial and Transverse Components
    \end{enumerate}
\end{enumerate}


\section*{Tutorials:}
There shall be related tutorials exercised in class and given as regular homework exercises. Tutorials can be as following for each specified chapters.

\begin{enumerate}
    \item \textbf{Introduction \hfill (1 hour)}
    \begin{itemize}
       \item[A.] Theory, definition and concept type questions
    \end{itemize}
    \item \textbf{Basic Concept in Statics and Static Equilibrium \hfill (2 hours)}
      \begin{itemize}
       \item[A.] Theory, definition and concept type questions
    \end{itemize}
    \item \textbf{Concept of Force acting on structures \hfill (3 hours)}
      \begin{itemize}
       \item[A.] Practical examples; numerical examples and derivation types of questions.
       \item[B.] There can be tutorials for each sub-section.
    \end{itemize}
    \item \textbf{Center of Gravity, Centroid and Moment of Inertia \hfill (4 hours)}
      \begin{itemize}
       \item[A.] Concept type; numerical examples and practical examples type questions.
    \end{itemize}
    
    \item \textbf{Friction \hfill (2 hours)}
      \begin{itemize}
       \item[A.] Definition type; Practical example type and numerical type questions.
    \end{itemize}
    
    \item \textbf{Analysis of Beam and Frame \hfill (5 hours)}
      \begin{itemize}
       \item[A.] Concept type; definition type; numerical examples type with diagrams questions.
       \item[B.] There can be tutorials for each sub-section.
    \end{itemize}
    
    \item \textbf{Analysis of Plane Trusses \hfill (5 hours)}
      \begin{itemize}
       \item[A.] Concept type; definition type; numerical examples type with questions.
       \item[B.] There can be tutorials for each sub-section.
    \end{itemize}
    
    \item \textbf{Kinematics of Particles and Rigid Body \hfill (4 hours)}
      \begin{itemize}
       \item[A.] Definition type; numerical examples type questions.
       \item[B.] There can be tutorials for each sub-section.
    \end{itemize}
    
    \item \textbf{Kinetics of Particles and Rigid Body: Force and Acceleration \hfill (4 hours)}
      \begin{itemize}
       \item[A.] Concept type; definition type; numerical examples type questions.
       \item[B.] There can be tutorials for each sub-section.
    \end{itemize}
\end{enumerate}


\section*{References:}
\begin{enumerate}
    \item F.P. Beer and E.R. Johnston, Jr. , ``Mechanics of Engineers- Statics and Dynamics", Mc Graw-Hill.
    
    \item R.C. Hibbeler, Ashok Gupta, ``Engineering Mechanics- Statics and Dynamics", New Delhi, Pearson.
    
    \item I.C. Jong and B.G. Rogers, ``Engineering Mechanics- Statics and Dynamics".
    
    \item D.K. Anand and P.F. Cunnif, ``Engineering Mechanics- Statics and Dynamics".
    
    \item R.S. Khurmi, ``A Text Book of Engineering Mechanics".
    
    \item R.S. Khurmi, ``Applied Mechanics and Strength of Materials".
    
    \item I.B. Prasad, ``A Text Book of Applied Mechanics".
    
    \item Shame, I.H., ``Engineering Mechanics - Statics and Dynamics", Prentice Hall of India, New Delhi.
\end{enumerate}