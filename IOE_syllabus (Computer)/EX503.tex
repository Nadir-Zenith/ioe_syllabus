\begin{center}
    \textbf{\huge{\uppercase{Electromagnetics}}}
    \\
    \vspace{.5cm}
    \textbf{\large{EX 502}}
\end{center}

\noindent\textbf{Lecture\ \ \ : 3} \hfill \textbf{Year : II } \\
\textbf{Tutorial \ : 1} \hfill \textbf{Part : I } \\
\textbf{Practical : 3/2}  \\

\par
\noindent 
\section*{Course Objective:}
To provide basic understanding of the fundamentals of Electromagnetics.

\mylist{
    \textbf{Introduction \hfill (3 hours)}
    \mylist{
        Co-ordinate system;
        Scalar and Vector fields;
        Operations on scalar and vector fields
    };
    \textbf{Electric field \hfill (12 hours)}
    \mylist{
        Coulomb's law;
        Electric field intensity;
        Electric flux density;
        Gauss's law and applications;
        Physical significance of divergence, Divergence theorem;
        Electric potential, potential gradient;
        Energy density in electrostatic field;
        Electric properties of material medium;
        Free and bound charge, polarization, relative permittivity, electric dipole;
        Electric Boundary conditions;
        Current, current density, conservation of charge, continuity equation, relaxation time;
        Boundary value problems, Laplace and Poisson equation and their solutions, uniqueness theorem;
        Graphical field plotting, numerical integration
    };
    \textbf{Magnetic field \hfill (9 hours)}
    \mylist{
        Biot-Savart's law;
        Magnetic field intensity;
        Ampere's circuital law and its application;
        Magnetic flux density;
        Physical significance of curl, Stoke's theorem;
        Scalar and magnetic vector potential;
        Magnetic properties of material medium;
        Magnetic force, magnetic torque, magnetic moment, magnetic dipole, magnetization;
        Magnetic boundary condition
    };
    \textbf{Wave equation and wave propagation \hfill (13 hours)}
    \mylist{
        Faraday's law, transformer emf, motional emf;
        Displacement current;
        Maxwell's equations in integral and point forms;
        Wave propagation in lossless and lossy dielectric;
        Plane waves in free space, lossless dielectric, good conductor;
        Power and pointing vector;
        Reflection of plane wave at normal incidence
    };
    \textbf{Transmission lines \hfill (5 hours)}
    \mylist{
        Transmission line equations;
        Input impedance, reflection coefficient, standing wave ratio;
        Impedance matching, quarter wave transformer, single stub matching, double stub matching
    };
    \textbf{Wave guides \hfill (2 hours)}
    \mylist{
        Rectangular wave guide;
        Transverse electric mode, transverse magnetic mode
    };
    \textbf{Antennas \hfill (1 hour)}
    \mylist{
        Introduction to antenna, antenna types and properties
    }
}

\section*{Practical:}
\mylist{
    Teledeltos (electro-conductive) paper mapping of electrostatic fields;
    Determination of dielectric constant, display of a magnetic Hysteresis loop;
    Studies of wave propagation on a lumped parameter transmission line;
    Microwave sources, detectors, transmission lines;
    Standing wave patterns on transmission lines, reflections, power patterns on transmission lines, reflections, power measurements;
    Magnetic field measurements in a static magnetic circuit, inductance, leakage flux
}

\section*{References:}
\mylist{
    W.H. Hayt, ``Engineering Electromagnetics", McGraw-Hill Book Company;
    J.D. Kraus, ``Electromagnetics", McGraw-Hill Book Company;
    N.N. Rao, ``Elements of Engineering Electromagnetics", Prentice Hall;
    Devid K. Cheng, ``Field and Wave Electromagnetics", Addison-Wesley;
    M.N. O. Sadiku, ``Elements of Electromagnetics", Oxford University Press
}