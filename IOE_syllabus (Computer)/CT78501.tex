\begin{center}
    \textbf{\huge{\uppercase{Remote Sensing}}}
    \\
    \vspace{.5cm}
    \textbf{\large{CT 785 01}}
\end{center}

\noindent\textbf{Lecture\ \ \ : 3} \hfill \textbf{Year : IV} \\
\textbf{Tutorial \ : 1} \hfill \textbf{Part : II } \\
\textbf{Practical : 3/2}  \\

\par
\noindent 
\section*{Course Objectives:}
To present an introduction to technological and scientific aspects of remote sensiong (RS) of the Earth and its atmosphere.


\mylist{
    \textbf{Introduction \hfill (7 hours)}
    \mylist{
        General concepts of remote sensing;
        History and basics of remote sensing of the Earth and its atmosphere;
        Classifications
    };
    \textbf{Physical Principles of Remote Sensing \hfill (10 hours)}
    \mylist{
        Basic quantities;
        Electromagnetic principles;
        Emission/radiation theory;
        Radar backscattering theory
    };
    \textbf{Remote Sensing Technology \hfill (12 hours)}
    \mylist{
        Passive remote sensing
        \mylist{
            Visible and infrared techniques;
            Microwave radiometry
        };
        Active remote sensing
        \mylist{
            Radar remote sensing;
            Lider remote sensing
        };
        Basics of satellite remote sensing, and ground truths
    };
    \textbf{Applications \hfill (10 hours)}
    \mylist{
        Earth and its atmosphere
        \mylist{
            Precipitation, winds, clouds and aerosols, temperature and trace gases;
            Vegetation, forestry, ecology;
            Urban and land use;
            Water planet: meteorological, oceanographic and hydrologic RS;
            Geological: Landforms, structure, topography, mine and resource exploration;
            Geographic information system (GIS): GIS approach to decision making
            Remote sensing into the 21st century: Outlook for the future RS
        }
    };
    \textbf{Remote Sensing Data \hfill (6 hours)}
    \mylist{
        Processing and classification of remote sensing data;
        Data formats;
        Retrieval algorithms;
        Analysis and image interpretations
    }
}


\section*{Practical:}
\mylist{
    Familiarization to remote sensing data available from department's capacity (via web and/or possible collaborations with national/international remote sensing agencies/institutions);
    Data visualization/graphics;
    Data processing and pattern recognition;
    Computer simulations;
    Technical Writing
}

\section*{References:}
\mylist{
    Campbell, J.B., ``Introduction to Remote sensing", The Guilford Press;
    Drury, S.A., ``Image Interpretation in Geology", Chapman \& Hall, 243 pp.;
    Drury, S.A., ``Images of the Earth: A Guide to Remote Sensing", Oxford press 212pp.;
    Kuehn, F.(Editor), ``Introductory Remote Sensing Principles and Concepts", Routledge, 215pp;
    Lillesand, T.M. and Kiefer, R. W., Remote Sensing and Image Interpretation", J. Wiley \& Sons, 720pp.;
    Sabins, Jr., F.F., ``Remote Sensing: Principles and Interpretation", W.H. Freeman \& Co., 496pp;
    Siegal, B.S. and Gillespie, A.R., ``Remote Sensing in Geology", J. Wiley and Sons (especially Chapter 1 through 11);
    Swain, P.H. and Davis, S.M., ``Remote sensing-the Quantitative Approach", McGraw Hill;
    Chen, H.S., ``Space Remote Sensing System: An Introduction", Academic press, Orlando;
    Jensen J.R., ``Remote sensing of the environment: An Earth resource perspective academic Press Orlando;
    Ulaby, F.T., R.K. Moore, and A. K. Fung, ``Microwave Remote Sensing Active and Passive;
    Periodicals devoted largely to remote sensing methods and applications.;
    IEEE Transactions on Geoscience and Report Sensing;
    IEEE Geoscience and Remote Sensing letters,
    International Journal of Remote Sensing;
    Photogrammetric Engineering and Remote Sensing;
    Remote Sensing of the Environments;
    Canadian Journal of Remote Sensing;
    Journal of Remote Sensing Society of Japan.
}