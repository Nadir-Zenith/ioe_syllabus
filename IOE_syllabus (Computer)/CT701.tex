\begin{center}
    \textbf{\huge{\uppercase{Project Management}}}
    \\
    \vspace{.5cm}
    \textbf{\large{CT 701}}
\end{center}

\noindent\textbf{Lecture\ \ \ : 3} \hfill \textbf{Year : IV} \\
\textbf{Tutorial \ : 1} \hfill \textbf{Part : I } \\
\textbf{Practical : 0}  \\

\par
\noindent 
\section*{Course Objective:}
To make the students able to plan, monitor and control project and project related activities.

\mylist{
    \textbf{Introduction \hfill (2 hours)} \\
    Definition of project and project management, Project objectives, classification of projects, project life cycle;
    \textbf{Project Management Body of Knowledge \hfill (4 hours)} \\
    Understanding of project environment, general management skill, effective and ineffective project managers, essential interpersonal and managerial skills, energized and initiator, communication, influencing, leadership, motivator, negotiation, problem solver, perspective nature, result oriented, global illiteracies, problem solving using problem trees.;
    \textbf{Portfolio and Project Management Institutes' (PMI) Framework \hfill (2 hours)} \\
    Portfolio, project management office, drivers of project success, inhibitors of project success;
    \textbf{Project Management \hfill (4 hours)} \\
    Advantages of project management, project management context as per PMI, Characteristics of project life cycles, representative project life cycles, IT Product Development Life Cycle, Product life cycle and project life cycle, System Development methodologies, role and responsibilities of key project members.;
    \textbf{Project and Organizational Structure \hfill (2 hours)} \\
    System view of project management, functional organization, matrix organization, organizational structure influences on projects.;
    \textbf{Project Management Process Groups \hfill (2 hours)} \\
    Project management processes, Overlaps of process groups in a phase, mapping of project management process groups to area of knowledge;
    \textbf{Project Integration Management \hfill (4 hours)} \\
    Develop project charters, develop preliminary project scope statement, Develop project management plan, direct and manage project execution, monitor and control project work, integrated change control, close project, project scope management, Create work break down structure, Scope verification, Scope control;
    \textbf{Project Time Management \hfill (4 hours)} \\
    Activity definition, decomposition of activities, activity attributes, Activity sequencing, precedence relationship, network diagram, precedence diagram method, arrow diagramming method, Activity resources estimating, determining resource requirements, Schedule development and control, principles of scheduling, milestones, forward pass, backward pass, critical path method, critical chain technique, gantt chart, schedule control.;
    \textbf{Project Cost Management \hfill (4 hours)} \\
    Cost and project, cost management, Cost estimating, types of cost estimates, estimating process and accuracy, enterprise environmental factors, organizational process assets, cost estimating tools, Cost budgeting, Cost aggregation, deriving budget from activity cost, Cost control process, cost control methods, earned value management, EVM benefits, variance analysis.;
    \textbf{Project quality management \hfill (3 hours)} \\
    Quality theories, Quality planning, project quality requirements, cost of quality, quality management plan, Quality assurance, quality audit, approach to a quality audit, Quality control process, control chart, pareto charts, testing of IT system, the test life cycle.;
    \textbf{Project Communication Management \hfill (3 hours)} \\
    Importance of communication management, Communications planning process, communication requirement analysis, organizing and conducting effective meeting, Information distribution process, Performance reporting process, integrated reporting system.;
    \textbf{Project Risk Management \hfill (4 hours)} \\
    Understanding Risk, project risk, Risk management planning process, risk management plan, risk identification, risk identification techniques, Qualitative risk analysis process, Quantitative risk analysis process, modeling techniques, Risk response planning, resolution of risk, strategies for negative risks or threats, strategies for positive risks or opportunities, Risk monitoring and control process.;
    \textbf{Project Procurement Management \hfill (3 hours)} \\
    Procurement management process flow, Plan purchases and acquisition process, enterprise environmental factor, organizational process assets, Plan contracting process, standard forms, evaluation criteria, Request seller response process, Select seller process, Contract administration process, Contract closure process.;
    \textbf{Developing Custom Processes for IT projects \hfill (3 hours)} \\
    Developing IT project management methodology, Moving forward with customized management processes, Certified associate in project management, Project management maturity, Promoting project Excellency through awards and assessment, Certification process flow, Code of ethics, Future trends.;
    \textbf{Balanced scorecard and ICT project management \hfill (1 hour)}
}

\section*{References:}
\mylist{
    M.C. Christensen and R.H. Thayer, ``The Project Manager's Guide to Software Engineering's Best Practices", IEEE computer society;
    Clifford F. Gray, Erik W. Larson, ``Project Management: The management of Process", McGraw Hill;
    Nick Jenkins,  ``A Project Management Primer";
    Trevor L. Young, ``A handbook of Project Management", Kogan Page India Pvt. Ltd.;
    M. Gentle, ``Balance Supply and Demand", Compuware;
    Kelkar, ``IT project Management"
}