\begin{center}
    \textbf{\huge{\uppercase{Engineering Chemistry}}}
    \\
    \vspace{.5cm}
    \textbf{\large{SH 453}}
\end{center}

\noindent\textbf{Lecture\ \ \ : 3} \hfill \textbf{Year : I } \\
\textbf{Tutorial \ : 1} \hfill \textbf{Part : II } \\
\textbf{Practical : 3}  \\

\par
\noindent 
\section*{Course Objective:}
To develop the basic concepts of Physical Chemistry, Inorganic Chemistry and Organic Chemistry relevant to problems in engineering.


\begin{enumerate}
    \item \textbf{Electro-chemistry and Buffer \hfill (6 hours)}
    \begin{enumerate}
        \item Electro-chemical cells
        \item Electrode Potential and Standard Electrode Potential 
        \item Measurement of Electrode Potential
        \item Nernst Equation
        \item EMF of Cell
        \item Application of Electrochemical and Electrolytic cells
        \item Electrochemical Series and its Application
        \item Buffer: its type and mechanism
        \item Henderson's equation for pH of buffer and related problems
        \item Corrosion and its type
        \item Factors influencing corrosion
        \item Prevention of corrosion
    \end{enumerate}
    
    \item \textbf{Catalyst \hfill (4 hours)}
    \begin{enumerate}
        \item Introduction
        \item Action of Catalyst (Catalytic Promoters and Catalytic Poisons)
        \item Characteristics of Catalyst
        \item Types of Catalyst
        \item Theories of Catalysis
        \item Industrial Applications of Catalysts
    \end{enumerate}
    
    \item \textbf{Environmental Chemistry \hfill (5 hours)}
    \begin{enumerate}
        \item Air Pollution
        \item Air pollutants i) gases $SO_x$, $NO_x$, $CO$, $CO_2$, $O_3$ and hydrocarbons \\
        ii) Particulates dust, smoke and fly ash
        \item Effects of Air Pollutants on human beings and their possible remedies
        \item Ozone depletion and its photochemistry
        \item Water Pollution (Ref of surface water and pound water)
        \item Water Pollutants (Ref of surface water) their adverse effect and remedies 
        \item Soil pollution
        \item Pollutants of soil their adverse effects and possible remedies
    \end{enumerate}
    
    \item \textbf{Engineering Polymers \hfill (6 hours)}
    \begin{enumerate}
        \item Inorganic polymers
        \item General properties of inorganic polymers
        \item Polyphosphazines
        \item Sulpher Based Polymers
        \item Chalcogenide Glasses
        \item Silicones
        \item Organic Polymers
        \item Types of Organic Polymers
        \item Preparation and application of \\ i) Polyurethane ii) Polystyrene \\ iii) Polyvinylchloride iv) Teflon \\ v)Nylon 6, 6 and vi) Bakelite \\ vii) Epoxy Resin vii) Fiber Reinforced Polymer
        \item Concept of bio-degradable, non-biodegradable and conducting polymers
    \end{enumerate}
    
    \item \textbf{3-d Transition elements and their applications \hfill (5 hours)}
    \begin{enumerate}
        \item Introduction
        \item Electronic Configuration
        \item Variable oxidation states
        \item Complex formation tendency
        \item Color formation
        \item Magnetic properties
        \item Alloy formation
        \item Applications of 3-d transition elements
    \end{enumerate}
    
    \item \textbf{Coordination Complexes \hfill (5 hours)}
    \begin{enumerate}
        \item Introduction
        \item Terms used in Coordination Complexes
        \item Werner's Theory Coordination Complexes
        \item Sidgwick's model and Sidgwick's effective atomic number rule
        \item Nomenclature of coordination compounds (Neutral type, simple cation and complex anion and complex cation and simple anion type)
        \item Valence Bond Theory of Complexes
        \item Application of valence bond theory in the formation of 
        \begin{itemize}
            \item [i)] Tetrahedral Complexes
            \item [ii)] Square planar Complexes and
            \item [iii)] Octahedral Complexes
        \end{itemize}
        \item Limitations of Valence Bond Theory
        \item Applications of Coordination Complexes
    \end{enumerate}
    
    \item \textbf{Explosives \hfill (3 hours)}
    \begin{enumerate}
        \item Introduction
        \item Type of explosives: Primary, Low an High explosives
        \item Preparation and application of TNT, TNG, Nitrocellulose and Plastic explosives
    \end{enumerate}
    
    \item \textbf{Lubricants and Paints \hfill (3 hours)}
    \begin{enumerate}
        \item Introduction
        \item Function of Lubricants
        \item Classification of Lubricants (Oils ,Greases and Solid)
        \item Paints
        \item Types of Paint
        \item Application of Paints
    \end{enumerate}
    
    \item \textbf{Stereochemistry \hfill (4 hours)}
    \begin{enumerate}
        \item Introduction
        \item Geometrical Isomerism (Cis Trans Isomerism) Z and E concept fo Geometrical Isomerism
        \item Optical Isomerism with references to two asymmetrical carbon center molecules 
        \item Terms Optical activity, Enantiomers, Diastereomers, Meso structures, Racemic mixture and Resolution
    \end{enumerate}
    
    \item \textbf{Reaction Mechanism in Organic reactions \hfill (4 hours)}
    \begin{enumerate}
        \item Substitution reaction
        \item Types of substitution reaction SN$^1$ and SN$^2$
        \item Elimination reaction
        \item Types of elimination reaction E1 and E2
        \item Factors governing SN$^1$, SN$^2$, E1 and E2 reaction mechanism path
    \end{enumerate}
\end{enumerate}

\section*{Practical:}
\begin{enumerate}
    \item Compare the alkalinity of different water samples by double indicator method \hfill 6 periods
    \item Determine the temporary and permanent hardness of water by EDTA Complexo-metric method \hfill 3 periods
    \item Determine residual and combined chlorine present in the chlorinated sample of water by Iodometric method \hfill 6 periods
    \item Prepare organic polymer nylon 6,6/ Bakelite in the laboratory \hfill 3 periods
    \item Determine the pH of different sample of buffer solution by universal indicator method \hfill 6 periods
    \item Prepare inorganic complex in the laboratory \hfill 3 periods
    \item Determine Surface tension of the given detergent solution and compare its cleansing power with other detergent solutions \hfill 6 periods
    \item Construct an electrochemical cell in the laboratory and measure the electrode potential of it \hfill 3 periods
    \item Estimate the amount of iron present in the supplied sample of ferrous salt using standard potassium permanganate solution (redox titration) \hfill 6 periods
\end{enumerate}


\section*{References:}
\begin{enumerate}
    \item Jain and Jain, ``Engineering Chemistry", Dhanpat Rai Publishing Co.
    \item Shashi Chawala, ``A Text Book of Engineering Chemistry", Dhanpat Rai Publishing Co.
    \item J.D. Lee, ``A New Concise Inorganic Chemistry", Wiley India Pvt. Limited.
    \item Marron and Prutton, ``Principles of Physical Chemistry", S. Macmillan and Co. Ltd.
    \item Bahl and Tuli, ``Essential of Physical Chemistry", S. Chand and Co. Ltd.
    \item Satya Prakash and Tuli, ``Advanced INorganic Chemistry Vol 1 and 2", S. Chand and Co. Ltd.
    \item Morrison and Boyd, ``Organic chemistry"
    \item Moti Kaji Sthapit, ``Selected Topics in Physical Chemistry", Taleju Prakashan, Kathmandu
    \item Peavy, Rowe and Tchobanoglous, ``Environmental Engineering", McGraw Hill, New York.
    \item R.K. Sharma, B. Panthi and Y. Gotame, ``Textbook of Engineering Chemistry", Athrai Publication
    
\end{enumerate}