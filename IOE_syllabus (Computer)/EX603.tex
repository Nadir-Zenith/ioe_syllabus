\begin{center}
    \textbf{\huge{\uppercase{Computer Graphics}}}
    \\
    \vspace{.5cm}
    \textbf{\large{EX 603}}
\end{center}

\noindent\textbf{Lecture\ \ \ : 3} \hfill \textbf{Year : III} \\
\textbf{Tutorial \ : 1} \hfill \textbf{Part : I } \\
\textbf{Practical : 3/2}  \\

\par
\noindent 
\section*{Course Objective:}
History of computer graphics, Applications of computer graphics, Hardware: Raster-Scan Displays, Vector Displays, Hard copy devices, Input hardware, display architectures, Applications in various fields like medicine, engineering, art, uses in virtual realism.

\mylist{
    \textbf{Introduction and application \hfill (2 hours)}
    \mylist{
    History of computer graphics; Applications of computer graphics; Hardware: Raster-Scan Displays, Vector Displays, Hard copy devices, Input hardware, display architectures; Applications in various fields like medicine, engineering, art, uses in virtual realism.
    };
    \textbf{Scan-Conversion \hfill (6 hours)}
    \mylist{
        Scan-Converting A point;
        Scan-Converting A straight line: DDA line Algorithm, Bresenham's Line Algorithm;
        Scan-Converting a Circle and an Ellipse: Mid-Point Circle and Ellipse algorithm;
    };
    \textbf{Two-dimensional Transformations \hfill (6 hours)}
    \mylist{
        Two-dimensional translation, rotation, scaling, reflection, shear transforms;
        Two-dimensional composite transformation;
        Two-dimensional viewing pipeline, world to screen viewing transformations and clipping (Cohen-Sutherland Line clipping, Liang-Barsky line clipping)
    };
    \textbf{Three-dimensional Graphics \hfill (6 hours)}
    \mylist{
        Three-dimensional translation, rotation, scaling, reflection, shear transforms;
        Three-dimensional composite transformation;
        Three-dimensional viewing pipeline, world to screen viewing transformations and clipping (orthographic, parallel, perspective projections)
    };
    \textbf{Curve Modeling \hfill (4 hours)}
    \mylist{
        Introduction to Parametric cubic curves, Splines, Bezier curves
    };
    \textbf{Surface Modeling \hfill (4 hours)}
    \mylist{
        Polygon surface, vertex table, edge table, polygon table, surface normal, and spatial orientation of surfaces
    };
    \textbf{Visible Surface Determination \hfill (6 hours)}
    \mylist{
        Image Space and Object Space techniques;
        Back Face Detection, Z-Buffer, A-Buffer, Scan-line method
    };
    \textbf{Illumination and Surface Rendering methods \hfill (8 hours)}
    \mylist{
        Algorithms to simulate ambient, diffuse and specular reflections;
        Constant, Gouraud and Phong shading models
    };
    \textbf{Introduction to OpenGL \hfill (3 hours)}
    \mylist{
        Introduction to OpenGL, callback functions, Color commands, drawing pixels, lines and polygons using OpenGL, Viewing, Lighting.
    }
}

\section*{Practical:}
There shall be 5 to 6 lab exercise including following concepts:
\mylist{
    DDA Line Algorithm;
    Bresenham's Line Algorithm;
    Mid-point Circle Algorithm;
    Mid-point Ellipse Algorithm;
    Lab on 2D Transformations;
    Basic Drawing Techniques in OpenGL
}


\section*{References:}
\mylist{
    Donald Hearn and M. Pauline Baker, ``Computer Graphics C version";
    Donald D. Hearn and M. Pauline Baker, ``Computer Graphics with OpenGL";
    Foley, Van Dam, Feiner, Hughes, ``Computer Graphics Principles and Practice"
}