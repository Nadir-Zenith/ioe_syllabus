\begin{center}
    \textbf{\huge{\uppercase{Electronic Devices and Circuits}}}
    \\
    \vspace{.5cm}
    \textbf{\large{EX 501}}
\end{center}

\noindent\textbf{Lecture\ \ \ : 3} \hfill \textbf{Year : II } \\
\textbf{Tutorial \ : 1} \hfill \textbf{Part : I } \\
\textbf{Practical : 3/2}  \\

\par
\noindent 
\section*{Course Objective:}
To introduce the fundamentals of analysis of electronic circuits and to provide basic understanding of semiconductor devices and analog integrated circuits.

\mylist{
    \textbf{Diodes \hfill (5 hours)}
    \mylist{
        The Ideal Diode;
        Terminal Characteristics of Junction Diodes;
        Physical Operation of Diodes;
        Analysis of Diode Circuits;
        Small Signal Model and Its Application;
        Operation in the Reverse Breakdown Region-- Zener Diodes
    };
    \textbf{The Bipolar Junction Transistor \hfill (10 hours)}
    \mylist{
        Operation of the npn transistor in the Active Mode;
        Graphical Representation of Transistor Characteristics;
        Analysis of Transistor Circuits at DC;
        Transistor as an amplifier;
        Small Signal equivalent circuit models;
        Graphical load line analysis;
        Biasing BJT for Discrete-Circuit Design;
        Basic Single-Stage BJT Amplifier Configurations (C-B, C-E, C-C);
        Transistor as a Switch -- Cutoff and Saturation;
        A General Large-Signal model for the BJT: The Ebers-Moll Model
    };
    \textbf{Field-Effect Transistor \hfill (9 hours)}
    \mylist{
        Structure and Physical Operation of Enhancement-Type MOSFET;
        Current-Voltage Characteristics of Enhancement-Type MOSFET;
        The Depletion-Type MOSFET;
        MOSFET Circuits at DC;
        MOSFET as an Amplifier;
        Biasing in MOS Amplifier Circuits;
        Junction Field-Effect Transistor
    };
    \textbf{Output Stages and Power Amplifiers \hfill (9 hours)}
    \mylist{
    Classification of Output Stages;
    Class A Output Stage;
    Class B Output Stage;
    Class AB Output Stage;
    Biasing the Class AB Stage;
    Power BJTs;
    Transformer-coupled Push-pull stages;
    Tuned Amplifiers
    };
    \textbf{Signal Generator and Waveform-Shaping Circuits (6 hours)}
    \mylist{
        Basic Principles of Sinusoidal Oscillator;
        Op Amp-RC Oscillator Circuits;
        LC and Crystal Oscillator;
        Generation of Square and Triangular Waveforms Using Astable Multivibrators;
        Integrated Ciruit Timers;
        Precision Rectifier Circuits
    };
    \textbf{Power Supplies, Breakdown Diodes, and Voltage Regulators \hfill (6 hours)}
    \mylist{
        Unregulated Power Supply;
        Bandgap Voltage Reference, a Constant Current Diodes;
        Transistor Series Regulators;
        Improving Regulator Performance;
        Current Limiting;
        Integrated Circuit Voltage Regulator
    }
}


\section*{Practical:}
\mylist{
    Bipolar Junction Transistor Characteristics and Single Stage Amplifier;
    Field-Effect Transistor Characteristics and Single Stage Amplifier;
    Power Amplifiers;
    Relaxation Oscillator and Sinusoidal Oscillator;
    Series and Shunt Voltage Regulators
}

\section*{References:}
\mylist{
    A.S. Sedra and K.C. Smith, ``Microelectronics Circuits", Oxford University Press;
    David A. Bell, ``Electronics Devices and Circuits", PHI;
    Robert Boylestad and Louis NashelSky, ``Electronic Device and Circuit Theory", PHI;
    Thomos L. Floyd, ``Electronic Devices", Pearson Education Inc.;
    Mark N. Horenstein, ``Microelectronic Circuits and Devices", PHI;
    Paul Horowitz and Winfield Fill, ``The Art of Elcetronics", Cambridge Publication;
    Jacob Millman and Christos C. Halkias, ans Satyabratajit, ``Millman's Elecronic Device and Circuits", Tata McGraw Hill
}