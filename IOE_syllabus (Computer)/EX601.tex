\begin{center}
    \textbf{\huge{\uppercase{Advanced Electronics}}}
    \\
    \vspace{.5cm}
    \textbf{\large{EX 601}}
\end{center}

\noindent\textbf{Lecture\ \ \ : 3} \hfill \textbf{Year : III} \\
\textbf{Tutorial \ : 1} \hfill \textbf{Part : I } \\
\textbf{Practical : 3/2}  \\

\par
\noindent 
\section*{Course Objective:}
To provide knowledge on data conversion, amplifiers, instrumentation and power circuits.

\mylist{
    \textbf{Operational Amplifier Circuits \hfill (6 hours)}
    \mylist{
        Bias circuits suitable for IC Design;
        The Widlar current source;
        The differential amplifier;
        Active loads;
        Output Stages
    };
    \textbf{Operational Amplifier Characterization \hfill (8 hours)}
    \mylist{
        Input offset voltage;
        Input bias and input offset currents;
        Output impedance;
        Differential and common-mode input impedance;
        DC gain, bandwidth, gain-bandwidth product;
        Common-mode and power supply rejection ratios;
        Higher frequency poles settling time;
        Slew rate;
        Noise in operational amplifier circuits
    };
    \textbf{Digital-To-Analog and Analog-To-Digital Conversion \hfill (8 hours)}
    \mylist{
        The R-2R ladder circuit;
        Unipolar and bipolar D/A converters;
        Count-up and tracking A/D's based on D/A's;
        Successive approximation A/D converters;
        Integrating voltage-to-time conversion A/D converters, dual and quad slop types;
        Sigma delta A/D converters;
        Flash A/D converters
    };
    \textbf{Instrumentation and Isolation Amplifiers \hfill (4 hours)}
    \mylist{
        One and two operational amplifier instrumentation amplifiers;
        The three operational amplifier instrumentation amplifiers;
        Consideration of non-ideal properties;
        Isolation amplifier principles and realization;
        Consideration of non-ideal properties
    };
    \textbf{Operational Amplifier-Bipolar Transistor Logarithmic Amplifier \hfill (3 hours)}
    \mylist{
        The basic logarithmic amplifier;
        Non-ideal effects;
        Stability consideration;
        Anti-logarithmic operations
    };
    \textbf{Log-Antilog Circuit Application \hfill (5 hours)}
    \mylist{
        Analog multiplier based on log-antilog principles;
        The multifunction converter circuit;
        Proportional to absolute temperature (PTAT) devices;
        RMS to dc conversion
    };
    \textbf{Introduction to Power Electronics \hfill (7 hours)}
    \mylist{
        Diodes, thyristors, triacs, IGBT;
        Controlled rectifier circuits;
        Inverters;
        Choppers;
        DC-to-DC conversion;
        AC-to-AC conversion
    };
    \textbf{Switched Power Supplies \hfill (4 hours)}
    \mylist{
        Voltage step-down regulators;
        Voltage step-up regulators;
        Step-up/step-down regulators;
        Filtering considerations;
        Control circuits, IC switched
    }
}

\section*{Practical:}
\mylist{
    Characteristics of operational amplifier;
    4 bit D to A converter;
    Differential amplifier, Instrumentation amplifier;
    Logarithmic amplifier;
    Study of switched voltage regulator;
    Study of Silicon-controlled-rectifier (SCR) and TRIAC circuit
}

\section*{Reference:}
\mylist{
    A.S. Sedra adn K.C. Smith, ``Microelectronic Circuits", Oxford University Press.;
    W. Stanely, ``Operational Amplifiers with Linear Integrated Circuits", Charles E. Merrill Publishing Company, Toronto;
    Jacob Millman and Christos C. Halkais, ``Integrated Electronics", TATA McGraw Hill Ediiton;
    Muhammad H. Rashid, ``Power Electronics: Circuits, Devices and Applications", Pearson Education.;
    Ramakant A. Gayakwad, ``Operational Amplifiers with Linear Integrated Circuits", Prentice Hall, New Delhi;
    Robert F. Coughlin and Frederick F. Driscoll, ``Operational Amplifiers and Linear Integrated Circuits", Prentice Hall, New Delhi;
    C.W. Lander, ``Power Electronics", McGraw Hill Book Company, New York;
    J.G. Graeme, ``Application of Operational Amplifiers: Third Generation Techniques", The Burr-Brown Electronics Series, McGraw Hill, New York;
    N. Mohan, T.M. Undeland and W.P. Robbins, ``Power Electronics Converters, Applications and Design", John Wiley and Sons, New York
}