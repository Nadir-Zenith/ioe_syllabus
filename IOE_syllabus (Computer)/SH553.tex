\begin{center}
    \textbf{\huge{\uppercase{Numerical Methods}}}
    \\
    \vspace{.5cm}
    \textbf{\large{SH 553}}
\end{center}

\noindent\textbf{Lecture\ \ \ : 3} \hfill \textbf{Year : II } \\
\textbf{Tutorial \ : 1} \hfill \textbf{Part : II } \\
\textbf{Practical : 3}  \\

\par
\noindent 
\section*{Course Objective:}
To introduce numerical methods used for the solution of engineering problems. The course emphasizes algorithm development and programming and application to realistic engineering problems.

\mylist{
    \textbf{Introduction, Approximation and errors of Computation \hfill (4 hours)}
    \mylist{
        Introduction, Importance of Numerical Methods;
        Approximation and Errors in computation;
        Taylor's series;
        Newton's Finite Differences(forward, Backward, central difference, divided difference);
        Difference operators, shift operators, differential operators;
        Uses and Importance of Computer programming in Numerical Methods
    };
    \textbf{Solutions of Nonlinear Equations \hfill (5 hours)}
    \mylist{
        Bisection Method;
        Newton Raphson method (two equation solution);
        Regula-False Method, Secant method;
        Fixed point iteration method;
        Rate of convergence and comparisons of these Methods
    };
    \textbf{Solution of system of linear algebraic equations \hfill (8 hours)}
    \mylist{
        Gauss elimination method with pivoting strategies;
        Gauss-Jordan method;
        LU Factorization;
        Iterative methods (Jacobi method, Gauss-Seidel method);
        Eigen value and Eigen vector using Power method
    };
    \textbf{Interpolation \hfill (8 hours)}
    \mylist{
        Newton's Interpolation (forward, backward);
        Central difference interpolation: Stirling's Formula, Bessel's Formula;
        Lagrange interpolation;
        Least square method of fitting linear and nonlinear curve for discrete data and continuous function;
        Spline Interpolation (Cubic Spline)
    };
    \textbf{Numerical Differentiation and Integration \hfill (6 hours)}
    \mylist{
        Numerical Differentiation formula;
        Maxima and Minima;
        Newton-Cote general quadrature formula;
        Trapezoidal, Simpson's 1/3, 3/8 rule;
        Romberg Integration;
        Gaussian integration (Gaussian -- Legendre formula 2 point and 3 points)
    };
    \textbf{Solution of ordinary differential equations \hfill (6 hours)}
    \mylist{
        Euler's and modified Euler's method;
        Runge Kutta methods for 1st and 2nd order ordinary differential equations;
        Solution of boundary value problem by finite difference method and shooting method
    };
    \textbf{Numerical solution of Partial differential equation \hfill (8 hours)}
    \mylist{
        Classification of partial differential equation(Elliptic, parabolic, and Hyperbolic);
        Solution of Laplace equation (Standard five point formula with iterative method);
        Solution of Poisson equation (finite difference approximation);
        Solution of Elliptic equation by Relaxation method;
        Solution of one dimensional Heat equation by Schmidt method
    }
}

\section*{Practical:}
Algorithm and program development in C programming language of following:
\mylist{
    Generate difference table;
    At least two from Bisection method, Newton Raphson method, Secant method;
    At least one from Gauss elimination method or Gauss Jordan method. Finding largest Eigen value and corresponding vector by Power method;
    Legrange interpolation. Curve fitting by least square method;
    Differentiation by Newton's finite difference method. Integration using Simpson's 3/8 rule;
    Solution of 1$^st$ order differential equation using RK-4 method;
    Partial differential equation (Laplace equation);
    Numerical solutions using MATLAB
}


\section*{References:}
\mylist{
    Dr. B.S. Grewal, ``Numerical Methods in Engineering and Science", Khanna Publication;
    Robert J. Schilling, Sandra Lharries, ``Applied Numerical Methods for Engineers using MATLAB and C", Thompson Brooks/cole.
    Richard L. Burden, J. Douglas Faires, "Numerical Analysis", Thomson Brooks/cole;
    John H. Mathews, Kurtis Fink, ``Numerical Methods Using MATLAB", Prentice Hall Publication;
    Jaan Kiusalaas, ``Numerical Methods in Engineering with MATLAB", Cambridge Publication
}