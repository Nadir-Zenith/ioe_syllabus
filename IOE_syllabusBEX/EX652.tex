\begin{center}
    \textbf{\huge{\uppercase{Communication System I}}}
    \\
    \vspace{.5cm}
    \textbf{\large{EX 652}}
\end{center}

\noindent\textbf{Lecture\ \ \ : 3} \hfill \textbf{Year : III} \\
\textbf{Tutorial \ : 1} \hfill \textbf{Part : II } \\
\textbf{Practical : 3/2}  \\

\par
\noindent 
\section*{Course Objective:}
To introduce the student to the principles and building blocks of analog communication systems.

\mylist{
    \textbf{Introduction \hfill (4 hours)}
    \mylist{
        Analog and Digital communication sources, transmitters, transmission channels and receivers;
        Noise , distortion and interference. Fundamental limitation due to noise, distortion and interference;
        Types and reasons for modulation
    };
    \textbf{Representation of signals and systems in communication \hfill (4 hours)}
    \mylist{
        Review of signals (types, mathematical representation and applications);
        Linear/non-linear, time variant/invariant systems. Impulse response and transfer function of a system. Properties of LTI systems;
        Low pass and band pass signals and systems, bandwidth of the system, distortionless transmission, the Hilbert transform and its properties;
        Complex envelops rectangular (in-phase and quadrature components) and polar representation of band pass band limited signals
    };
    \textbf{Spectral Analysis \hfill (4 hours)}
    \mylist{
        Review of Fourier series and transform, energy and power, Parseval's theorem;
        Energy Density Spectrum, periodogram, power spectral density function (psdf);
        Power spectral density functions of harmonic signal and white noise;
        The autocorrelation (AC) function, relationship between psdf and AC function
    };
    \textbf{Amplitude Modulation \hfill (12 hours)}
    \mylist{
        Time domain expressions, frequency domain representation, modulation index, signal bandwidth;
        AM for a single tone message, carrier and side-band components, powers in carrier and side-band components bandwith and power efficiency;
        Generation of DSB-FC AM;
        Double Side Band Suppressed Carrier AM (DSB-AM), time and frequency domain expressions, powers in side-bands, bandwidth and power efficiency;
        Generation of DSB-AM (balanced, ring modulators);
        Single Side Band Modulation, time and frequency domain expressions, powers;
        Generation of SSB (SSB filters and indirect method);
        Vestigial Side Bands (VSB), Independent Side Bands (ISB) and Quadrature Amplitude Modulations (QAM)
    };
    \textbf{Demodulation of AM signals \hfill (6 hours)}
    \mylist{
        Demodulation of DSB-FC, DSB-SC and SSB using synchronous detection;
        Square law and envelop detection of DSB-FC;
        Demodulation of SSB using carrier reinsertion, carrier recovery circuits;
        Phase Locked Loop (PLL), basic concept, definitions, equations and applications, demodulation of AM using PLL.
    };
    \textbf{Frequency Modulation (FM) and Phase Modulation (PM) \hfill (12 hours)}
    \mylist{
        Basic definitions, time domain expressions for FM and PM;
        Time domain expression for single tone modulated FM signals, spectral representation, Bessel's functions;
        Bandwidth of FM, Carson's rule, narrow and wideband FM;
        Generation of FM (direct and Armstrong's methods);
        Demodulation of FM and PM signals, synchronous (PLL) and non-synchronous (limiter-discriminator) demodulation;
        Stereo FM, spectral detail, encoder and decoder;
        Pre-emphasis and de-emphasis networks;
        The superheterodyne radio receivers for AM and FM
    };
    \textbf{Frequency Division Multiplexing (FDM) \hfill (3 hours)}
    \mylist{
        Principle of frequency division multiplexing, FDM in telephony, hierarchy;
        Frequency Division Multiple Access (FDMA) systems - SCPC, DAMA, SPADE, etc.;
        Filter and oscillator requirements in FDM.
    }
}

\section*{Practical:}
\mylist{
    Demonstration of power spectrum of various signals using LF spectrum analyzer;
    Generation of DSB-SC, DSB-Fc and SSB signals;
    Demodulation of AM signals (Synchronous and non-synchronous methods);
    Generation of FM signals;
    Demodulation of FM signal (limiter-discriminator);
    Operation of PLL, PLL as demodulator of AM and FM signals.
}


\section*{References:}
\mylist{
    S. Haykin, Analog and Digital communication systems;
    Leon Couch, Digital and analog communication systems;
    B.P. Lathi, Analog and Digital communication systems;
    J. Proakis, Analog and Digital communication systems;
    D. Sharma, Course manual ``Communication Systems I"
}