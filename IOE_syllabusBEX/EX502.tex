\begin{center}
    \textbf{\huge{\uppercase{Digital Logic}}}
    \\
    \vspace{.5cm}
    \textbf{\large{EX 502}}
\end{center}

\noindent\textbf{Lecture\ \ \ : 3} \hfill \textbf{Year : II } \\
\textbf{Tutorial \ : 0} \hfill \textbf{Part : I } \\
\textbf{Practical : 3}  \\

\par
\noindent 
\section*{Course Objective:}
To introduce basic principles of digital logic design, its implementation and applications

\mylist{
    \textbf{Introduction \hfill (3 hours)}
    \mylist{
        Definitions for Digital Signals;
        Digital Waveforms;
        Digital Logic;
        Moving and Storing Digital Information;
        Digital Operations;
        Digital Computers;
        Digital Integrated Circuits;
        Digital IC Signal Levels;
        Clock wave form;
        Coding
        \mylist{
            ASCII Code;
            BCD;
            The Excess-3 code;
            The Gray Code
        }
    };
    \textbf{Digital Logic \hfill (1 hour)}
    \mylist{
        The Basic Gates -- NOT, OR, AND;
        Universal Logic Gates -- NOR, NAND;
        AND-OR-INVERT Gates;
        Positive and Negative Logic;
        Introduction to HDL
    };
    \textbf{Combinational Logic Circuits \hfill (5 hours)}
    \mylist{
        Boolean Laws and Theorems;
        Sum-of-products method;
        Truth table to Karnaugh Map;
        Pairs, Quads, and Octets;
        Karnaugh Simplifications;
        Don't Care conditions;
        Product-of-Sums Method;
        Product-of-Sums Simplification;
        Harards and Hazard Covers;
        HDL Implementation Models
    };
    \textbf{Data Processing Circuits \hfill (5 hours)}
    \mylist{
        Multiplexer;
        DeMultiplexer;
        Decoder;
        BCD-to-Decimal Decoders;
        Seven-segment decoders;
        Encoder;
        Exclusive-OR Gates;
        Parity Generators and Checkers;
        Magnitude Comparator;
        Read-Only Memory;
        Programmable Array Logic;
        Programmable Logic Arrays;
        Troubleshooting with a logic probe;
        HDL implementation of Data Processing Circuits
    };
    \textbf{Arithmetic Circuits \hfill (5 hours)}
    \mylist{
        Binary Addition;
        Binary Subtraction;
        Unsigned Binary Numbers;
        Sign-Magnitude Numbers;
        2's Complement Representation;
        2's Complement Arithmetic;
        Arithmetic Building Blocks;
        The Adder-Subtracter;
        Fast-Adder;
        Arithmetic logic unit;
        Binary Multiplication and Division;
        Arithmetic Circuits Using HDL
    };
    \textbf{Flip Flops \hfill (5 hours)}
    \mylist{
        RS Flip-Flops;
        Gated Flip-Flops;
        Edge-Triggered RS Flip-Flops;
        Edge-Triggered D Flip-Flops;
        Edge-Triggered JK Flip-Flops;
        Flip-Flop Timing;
        JK Master-Slave Flip-Flops;
        Switch Contacts Bounds Circuits;
        Various Representation of Flip-Flops;
        Analysis of Sequential Circuits
    };
    \textbf{Registers \hfill (2 hours)}
    \mylist{
        Types of Registers;
        Serial In -- Serial Out;
        Serial In -- Parallel Out;
        Parallel In -- Serial Out;
        Parallel In -- Parallel Out;
        Applications of Shift Registers
    };
    \textbf{Counters \hfill (5 hours)}
    \mylist{
        Asynchronous Counters;
        Decoding Gates;
        Synchronous Counters;
        Changing the Counter Modulus;
        Decade Counters;
        Counter Design as a Synthesis Problem;
        A Digital Clock
    };
    \textbf{Sequential Machines \hfill (8 hours)}
    \mylist{
        Synchronous machines
        \mylist{
            Clock driven models and state diagrams;
            Transition tables, Redundant states;
            Binary assignment;
            Use of flip-flops in realizing the models
        };
        Asynchronous machines
        \mylist{
            Hazards in asynchronous system and use of redundant branch;
            Allowable transitions;
            Flow tables and merger diagrams;
            Excitation maps and realization of the models
        }
    };
    \textbf{Digital Integrate Circuits \hfill (4 hours)}
    \mylist{
        Switching Circuits;
        7400 TTl;
        TTL parameters;
        TTL Overview;
        Open Collector Gates;
        Three-state TTL Devices;
        External Drive for TTL Loads;
        TTL Driving External Loads;
        74C00 CMOS;
        CMOS Characteristics;
        TTL-to-CMOS Interface;
        CMOS-to-TTL Interface
    };
    \textbf{Applications \hfill (2 hours)}
    \mylist{
        Multiplexing Displays;
        Frequency Counters;
        Time Measurement
    }
}


\section*{Practical:}
\mylist{
    DeMorgan's law and it's familiarization with NAND and NOR gates;
    Encoder, Decoder and Multiplexer;
    Familiarization with Binary Addition and Subtraction;
    Construction of true complement generator;
    Latches, RS, Master-Slave and T type flip flops;
    D and JK type flip flops;
    Ripple Counter, Synchronous counter;
    Familiarization with computer package for logic circuit design;
    Design digital circuits using hardware and software tools;
    Use of PLAs and PLDs
}

\section*{References:}
\mylist{
    Donald P. Leach, Albert Paul Malvino and Goutam Saha, ``Digital Principles and Applications", Tata McGraw-Hill;
    David J Comer, ``Digital Logic and State Machine Design", Oxford University Press;
    William I. Fletcher, ``An Engineering Approach to Digital Design", Prentice Hall of India, New Delhi;
    William H. Gothmann, ``Digital Electronics, An Introduction to Theory and Practice"
}