\begin{center}
    \textbf{\huge{\uppercase{Instrumentation I}}}
    \\
    \vspace{.5cm}
    \textbf{\large{EE 552}}
\end{center}

\noindent\textbf{Lecture\ \ \ : 3} \hfill \textbf{Year : II } \\
\textbf{Tutorial \ : 1} \hfill \textbf{Part : II } \\
\textbf{Practical : 3/2}  \\

\par
\noindent 
\section*{Course Objective:}
To provide comprehensive treatment of methods and instrument for a wide range of measurement problems.


\mylist{
    \textbf{Instrumentation Systems \hfill (2 hours)}
    \mylist{
        Functions of components of instrumentation system introduction, signal processing, signal transmission, output indication;
        Need for electrical, electronics, pneumatic and hydraulic working media systems and conversion devices;
        Analog and digital systems
    };
    \textbf{Theory of Measurement \hfill (10 hours)}
    \mylist{
        Static performance parameters -- accuracy, precision, sensitivity, resolution and linearity;
        Dynamic performance parameter -- response time, frequency response and bandwidth;
        Error in measurement;
        Statistical analysis of error in measurement;
        Measurement of voltage and current (moving coil and moving iron instruments);
        Measurement of low, high and medium resistances;
        AC bridge and measurement of inductance and capacitance
    };
    \textbf{Transducer \hfill (8 hours)}
    \mylist{
        Introduction;
        Classification;
        Application
        \mylist{
            Measurement of mechanical variables, displacement, strain, velocity, acceleration and vibration;
            Measurement of process variables -- temperature, pressure, level, fluid flow, chemical constituents in gases or liquids, pH and humidity;
            Measurement of bio-physical variables, blood pressure and myoelectric potentials
        }
    };
    \textbf{Electrical Signal Processing and transmission \hfill (6 hours)}
    \mylist{
        Basic Op-amp characteristics;
        Instrumentation amplifier;
        Signal amplification, attenuation, integration, differentiation, network isolation, wave shaping;
        Effect of noise, analog filtering, digital filtering;
        Optical communication,, fiber optics, electro-optic conversion devices
    };
    \textbf{Analog-Digital and Digital-Analog Conversion \hfill (6 hours)}
    \mylist{
        Analog signal and digital signal;
        Digital to analog converters -- weighted resistor type, R-2R ladder type, DAC errors;
        Analog to digital converters -- successive approximation type, ramp type, dual ramp type, flash type, ADC errors 
    };
    \textbf{Digital Instrumentation \hfill (5 hours)}
    \mylist{
        Sample data system, sample and hold circuit;
        Components of data acquisition system;
        Interfacing to the computer
    };
    \textbf{Electrical equipment \hfill (8 hours)}
    \mylist{
        Wattmeter
        \mylist{
            Types;
            Working principles
        };
        Energy meter
         \mylist{
            Types;
            Working principles
        };
        Frequency meter
         \mylist{
            Types;
            Working principles
        };
        Power factor meter;
        Instrument transformers
    }
}


\section*{Practical:}
\mylist{
    Accuracy test in analog meters;
    Operational Amplifiers in Circuits \\ -- Use of Op amp as a summer, inverter, integrator and differentiator;
    Use resistive, inductive and capacitive transducers to measure displacement \\ -- Use strain gauge transducers to measure force;
    Study of Various transducers for measurement of Angular displacement, Angular velocity, pressure and flow
    \begin{itemize}
    \item[--] Use optical, Hall effect and inductive transducer to measure angular displacement
    \item[--] Use tacho-generator to measure angular velocity
    \item[--] Use RTD transducers to measure pressure and flow
    \end{itemize};
    Digital to Analog Conversion \\ -- Perform static testing of D/A converter;
    Analog to Digital Conversion \\ -- Perform static testing of A/D converter
}

\section*{References:}
\mylist{
    D.M. Considine, ``Process Instruments and Controls Handbook", McGraw Hill;
    S. Wolf and R.F.M. Smith, ``Students Reference Manual for Electronics Instrumentation Laboratories", Prentice Hall;
    E.O. Deobelin, ``Measurement System, Application and Design", McGraw Hill;
    A.K. Sawhney, ``A Course in Electronic Measurement and Instrumentation", DhanpatRai and Sons;
    C.S. Rangan, G.R. Sharma and V.S.V. Mani, ``Instrumentation Devices and Systems", Tata McGraw Hill;
    J.B. Gupta, ``A course in Electrical and Electronics Measurement and Instrumentation" , Kataria and Sons
}