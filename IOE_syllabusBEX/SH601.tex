\begin{center}
    \textbf{\huge{\uppercase{Communication English}}}
    \\
    \vspace{.5cm}
    \textbf{\large{SH 601}}
\end{center}

\noindent\textbf{Lecture\ \ \ : 3} \hfill \textbf{Year : III} \\
\textbf{Tutorial \ : 1} \hfill \textbf{Part : I } \\
\textbf{Practical : 2}  \\

\par
\noindent 
\section*{Course Introduction:}
This course is designed for the students of engineering with the objective of developing all four skills of communication applicable in professional field.

\section*{Course Objective:}
After completion of this course students will be able to:
\begin{enumerate}[label=\alph*.]
    \item comprehend reading materials both technical and semi-technical in nature
    \item develop grammatical competence
    \item write notice, agenda, minutes
    \item write proposals
    \item write reports
    \item write research articles
    \item listen and follow instruction, description and conversation in native speakers' accent
    \item do discussion in group, deliver talk and present brief oral reports
\end{enumerate}


\subsection*{Unit I: Reading \hfill (15 hours)}

\mylist{
    \textbf{Intensive Reading \hfill (8 hours)}
    \mylist{
        Comprehension;
        Note-taking;
        Summary writing;
        Contextual questions based on facts and imagination;
        Interpreting text
    };
    \textbf{Extensive Reading \hfill (5 hours)}
    \mylist{
        Title/Topic Speculation;
        Finding theme;
        Sketching Character
    };
    \textbf{Contextual Grammar \hfill (2 hours)}
    \mylist{
        Sequence of tense;
        Voice;
        Subject-verb agreement;
        Conditional Sentences;
        Preposition
    }
}

\subsection*{Unit II: Introduction to technical writing process and meeting \hfill (4 hours)}
\mylist{
    \textbf{Editing, MLA/APA \hfill (2 hours)}
    \mylist{
        Composing and editing strategies;
        MLA and APA comparision
    };
    \textbf{Writing notices with agenda and minutes \hfill (2 hours)}
    \mylist{
        Introduction;
        Purpose;
        Process
    }
}

\subsection*{Unit III: Writing Proposal \hfill (6 hours)}
\mylist{
    Introduction
    \mylist{
        Parts of the proposal
        \mylist{
            Title page;
            Abstract/Summary;
            Statement of Problem;
            Rationale;
            Objectives;
            Procedure/Methodology;
            Cost estimate or Budget;
            Time management/Schedule;
            Summary;
            Conclusion;
            Evaluation or Follow-up;
            Works cited
        }
    }
}


\subsection*{Unit IV: Reports \hfill (18 hours)}
\mylist{
    \textbf{Informal Reports \hfill (6 hours)}
    \mylist{
        Memo report
        \mylist{
            Introduction;
            Parts
        };
        Letter report
        \mylist{
            Introduction;
            Parts
        }
    };
    \textbf{Project/Field Report \hfill (3 hours)}
    \mylist{
             Introduction;
            Parts
    };
    \textbf{Formal report \hfill (9 hours)}
    \mylist{
        Introduction;
        Types of formal reports
        \mylist{
            Progress report;
            Empirical/Research report;
            Technical report
        };
        Parts and Components of formal report
        \mylist{
            Preliminary section
            \mylist{
                Cover page;
                Letter of transmittal/Preface;
                Title page;
                Acknowledgements;
                Table of contents;
                List of figures and tables;
                Abstract/Executive summary
            };
            Main Section
            \mylist{
                Introduction;
                Discussion/Body;
                Summary/Conclusion;
                Recommendations
            };
            Documentation
            \mylist{
                Notes(Contextual/foot notes);
                Bibliography;
                Appendix
            }
        }
    }
}


\subsection*{Unit V: Writing Research Articles \hfill (2 hours)}
\mylist{
    Introduction;
    Procedures
}

\section*{Practical:}
\begin{table}[h]
    \centering
    \begin{tabular}{|c|m{30em}|c|}
    \hline
        \multicolumn{2}{|l|}{\textbf{Language Lab}} & 30 hours  \\
    \hline
        \multicolumn{2}{|l|}{\textbf{Unit I: Listening}} & 12 hours \\
    \hline
        \textbf{Activity I} & General instruction on effective listening, factors influencing listening, and note-taking to ensure attention. (Equipment required: Laptop, multimedia, laser pointer, Overhead projector, power point, DVD, video set, screen) & 2 hours \\
    \hline
     \textbf{Activity II} & Listening to recorded authentic instruction followed by exercises  (Equipment Required : Cassette player or laptop) & 2 hours \\
    \hline
     \textbf{Activity III} & Listening to recorded authentic description followed by exercises  (Equipment Required : Cassette player or laptop) & 4 hours \\
    \hline
     \textbf{Activity II} & Listening to recorded authentic conversation followed by exercises  (Equipment Required : Cassette player or laptop) & 4 hours \\
    
    \hline
    \multicolumn{2}{|l|}{\textbf{Unit II: Speaking}} & 18 hours \\
    \hline
    \textbf{Activity I} & General instruction on effective speaking ensuring audience's attention, comprehension and efficient use of Audio-visual aids. (Equipment required: Laptop, multimedia, laser pointer, Overhead projector, power point, DVD, video, screen) & 2 hours \\
    \hline
    \textbf{Activity II} & Making students express their individual views on the assigned topics (Equipment Required : Microphone, movie camera) & 2 hours \\
    \hline
    \textbf{Activity III} & Getting students to participate in group discussion on the assigned topics & 4 hours \\
    \hline
    \textbf{Activity IV} & Making students deliver talk either individually or in group on the assigned topics (Equipment Required: Overhead projector, microphone, power point, laser pointer, multimedia, video camera, screen) & 8 hours \\
    \hline
    \textbf{Activity V} & Getting Students to present their brief oral reports individually on the topics of their choice. (Equipment Required: Overhead projector, microphone, power point, laser pointer, multimedia, video camera, screen) & 2 hours \\
    \hline
    \end{tabular}
 
   
\end{table}
\newpage

\section*{Evaluation Scheme}

\begin{table}[h]
    \centering
    \begin{tabular}{|m{2em}|m{4em}|m{4em}|m{5em}|m{8em}|m{3em}|m{10em}|}
    \hline 
    Units & Testing Items & No. of Questions & Type of Questions & Marks Distribution & Total Marks & Remarks \\
    \hline
    I & Reading & 3 & For grammar: Objective and for the rest: Short & 2 Short question (5 + 5), Interpretation of text (5), Note + Summary (5 + 5), Grammar (5) & 30 & For short questions 2 to be done out of 3 from the seen passages, for interpretation an unseen paragraph of about 75 words to be given, for note and summary an unseen text of about 200 to 250 to be given, for grammar 5 questions of fill up the gaps or transformation type to be given \\
    \hline
    II & Introduction to technical writing process and meeting & 3 & MLA/APA : Objective, Editing and Meeting : Short & MLA/APA (4), Editing (5), Meeting (5) & 14 & For APA/MLA 4 questions to be given to transform one from another or 4 questions asking to show citation according to APA/MLA technique, For meeting minute alone or notice with agendas to be given \\
    \hline
    III & Proposal Writing & 1 & Long & 10 & 10 & A question asking to write a very brief proposal on any techincal topic to be given \\
    \hline
    IV & Report Writing & 2 & Informal report : Short, Formal report: Long & Informal report (6), Formal report (10) & 16 & A question asking to write very brief informal report on technical topic to be given, for formal report a question asking to write in detail on any three elements of a formal report on technical topic to be given. \\
    \hline
    V & Research article & 1 & Long & 10 & 10 & A question asking to write a brief research article on technical topic to be given \\
    \hline 
    \end{tabular}
 \end{table}
 
 
 \newpage
\section*{Evaluation Scheme for Lab}
\begin{table}[h]
     \begin{tabular}{|m{2em}|m{5em}|m{4em}|m{5em}|m{8em}|m{3em}|m{10em}|}
    \hline 
    Units & Testing Items & No. of Questions & Type of Questions & Marks Distribution & Total Marks & Remarks \\
    \hline
    I & Listening : instruction, Description, conversation & 2 & Objective  & 5 + 5 & 10 & Listening tape to be played on any two out of instruction, description and conversation followed by 10 multiple choice type or fill in the gaps type questions \\ 
    \hline
    II & Speaking : Group/round table discussion, presenting brief oral report, delivering talk & 2 & Subjective & Round table discussion (5), talk or brief oral report (10) & 15 & Different topics to be assigned in groups consisting of 8 members for group discussion and to be judged individually, individual presentation to be judged through either by talk on assigned topics or by brief oral reports based on their previous project, study and field visit. \\
    \hline
    
    \end{tabular}
    
\end{table}

\section*{Prescribed books:}
\mylist{
    Adhikari, Usha, Yadav, Rajkumar, Yadav, Bijaya {;} ``A course book of communicative English", Trinity publication;
    Adhikari, Usha, Yadav, Rajkumar, Yadav, Bijaya {;} ``Technical Communication English", Trinity publication;
    Khanal, Ramnath,``Need-based Language Teaching (Analysis in Relation to Teaching of English for Profession Oriented Learners)", Kathmandu: D. Khanal.;
    Konar, Nira, ``Communication Skills for Professional", PHI learning pvt. ltd.;
    Kumar, Ranjit, ``Research Methodology", Pearson Education;
    Laxminarayan, K.R., ``English for Technical Communication", Chennai, Scitech Publication Pvt. Ltd;
    Mishra, Sunita et. al., ``Communication Skills for Engineers", Pearson Education;
    Prasad, P. et. al., ``The functional Aspects of Communication skills", S.K., Kataria and Sons.;
    Rutherfoord, Andrea J. Phd, ``Basic Communication Skills for Technology", Pearson Education Asia;
    Rizvi, M. Ashraf, ``Effective Technical Communication", Tata McGraw Hill;
    Reinking A James et. al., ``Strategies for Successful Writing: A rhetoric, reshearch guide, reader and handbook", Prentice Hall, New Jersey;
    Sharma R.C. et. al.,``Business correspondence and Report Writing: A practical Approach to Business and Technical communication", Tata McGraw Hill;
    Sharma, Sangeeta et. al, ``Communication skills for Engineers and Scientists", PHI Learning Pvt. Ltd, New Delhi;
    Taylor, Shirley et. al, ``Model Business letters, E-mails and other Business documents", Pearson Education
}