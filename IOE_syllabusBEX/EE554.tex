\begin{center}
    \textbf{\huge{\uppercase{Electrical Machines}}}
    \\
    \vspace{.5cm}
    \textbf{\large{EE 554}}
\end{center}

\noindent\textbf{Lecture\ \ \ : 3} \hfill \textbf{Year : II } \\
\textbf{Tutorial \ : 1} \hfill \textbf{Part : II } \\
\textbf{Practical : 3/2}  \\

\par
\noindent 
\section*{Course Objective:}
To impart knowledge on constructional details, operating principle and performance of Transformers, DC Machines, 1-phase and 3-phase Induction Machines, 3-phase Synchronous Machines and Fractional Kilowatt Motors.

\mylist{
    \textbf{Magnetic Circuits and Induction \hfill (4 hours)}
    \mylist{
        Magnetic Circuits;
        Ohm's Law for Magnetic Circuits;
        Series and Parallel Magnetic circuits;
        Core with air gap;
        B-H relationship (Magnetization Characteristics);
        Hysteresis with DC and AC excitation;
        Hysteresis Loss and Eddy Current Loss;
        Faraday's Law of Electromagnetic Induction, Statically and Dynamically induced EMF;
        Force on Current Carrying Conductor
    };
    \textbf{Transformer \hfill (8 hours)}
    \mylist{
        Constructional Details, recent trends;
        Working principle and EMF equation;
        Ideal Transformer;
        No load and load Operation;
        Operation of Transformer with load;
        Equivalent Circuits and Phasor Diagram;
        Tests: Polarity test, Open Circuit test, Short Circuit test and Equivalent Circuit Parameters;
        Voltage Regulation;
        Losses in a transformer;
        Efficiency, condition for maximum efficiency and all day efficiency;
        Instrument Transformers: Potential Transformer (PT) and Current transformer (CT);
        Auto transformer: construction, working principle and Cu saving;
        Three phase Transformers
    };
    \textbf{DC Generator \hfill (6 hours)}
    \mylist{
        Constructional Details and Armature Winding;
        Working principle and Commutator Action;
        EMF equation;
        Method of excitation: separately and self excited, Types of DC Generator;
        Characteristics of series, shunt and compound generator;
        Losses in DC generators;
        Efficiency and Voltage Regulation
    };
    \textbf{DC Motor \hfill (6 hours)}
    \mylist{
        Working principle and Torque equation;
        Back EMF;
        Method of excitation, Types of DC motor;
        Performance Characteristics of DC motors;
        Starting of DC motors: 3 point and 4 point starters;
        Speed control of DC motors: Field control, Armature Control;
        Losses and Efficiency
    };
    \textbf{Three Phase Induction Machines \hfill (7 hours)}
    \mylist{
        Three phase induction motor
        \mylist{
            Constructional Details and Types;
            Operating Principle, Rotating Magnetic Field, Synchronous Speed, Slip, Induced EMF, Rotor current and its frequency, Torque Equation;
            Torque-Slip characteristics
        };
        Three phase induction generator
        \mylist{
            Working Principle, voltage build up in an Induction generator;
            Power stages
        }
    };
    \textbf{Three Phase Synchronous Machines \hfill (6 hours)}
    \mylist{
        Three Phase Synchronous Generator
        \mylist{
            Constructional Details, Armature Winding, Types of Rotor, Exciter;
            Working Principle;
            EMF equation, distribution factor, pitch factor;
            Armature Reaction and its effects;
            Alternator with load and its phasor diagram
        };
        Threee Phase Synchronous Motor
        \mylist{
            Principle of operation;
            Starting methods;
            No load and load operation, Phasor Diagram;
            Effect of Excitation and power factor control
        }
    };
    \textbf{Fractional Kilowatt Motors \hfill (6 hours)}
    \mylist{
        Single phase Induction Motors: Construction and Characteristics;
        Double Field Revolving Theory;
        Split phase Induction Motor
        \mylist{
            Capacitors start and run motor;
            Reluctance start motor
        };
        Alternating Current Series motor and Universal Motor;
        Special Purpose Machines: Stepper motor, Schrage motor and Servo motor
    }
}


\section*{Practical:}
\mylist{
    Magnetic Circuits \\ - To draw B-H curve for two different sample of Iron Core \\ - Compare their relative permeability;
    Two Winding Transformers \\ - To perform turn ratio test \\ - To perform open circuit (OC) and short circuit (SC) test to determine equivalent circuit parameter of a transformer and hence to determine the regulation and efficiency at full load;
    DC Generator \\ - To draw open circuit characteristics (OCC) of a DC shunt generator \\ - To draw load characteristic of shunt generator;
    DC Motor \\ - Speed control of DC Shunt motor by (a) armature control method (b) field control method \\ - To observe the effect of increasing load on DC shunt motor's speed, armature current, and field current;
    3-phase Machines \\ - To draw torque-speed characteristics and to observe the effect of rotor resistance on torque-speed characteristics of a 3-phase Induction Motor \\ - To study load characteristics of synchronous generator with (a) resistive load (b) inductive load and (c) capacitive load;
    Fractional Kilowatt Motors \\ - To study the effect of a capacitor on the starting and running of a single-phase induction motor \\ - Reversing the direction of rotation of a single phase capacitor induct
}


\section*{References:}
\mylist{
    I.J. Nagrath and D.P. Kothari, ``Electrical Machines", Tata McGraw Hill;
    S.K. Bhattacharya, ``Electrical Machines", Tata McGraw Hill;
    B.L. Theraja and A.K. Theraja, ``Electrical Technology (Vol-II)", S. Chand;
    Husain Ashfaq, ``Electrical Machines", Dhanpat Rai and Sons;
    A.E. Fitzgerald, C. Kingsley Jr. and Stephen D. Umans, ``Electic Machinery", Tata McGraw Hill;
    B.R. Gupta and Vandana Singhal, ``Fundamentals of Electrical Machines", New Age International;
    P.S. Bhimbra, ``Electrical Machines", Khanna Publishers;
    Irving L. Kosow, ``Electric Machine and Transformers", Prentice Hall of India;
    M.G. Say, ``The Performance and Design of AC machines", Pit man \& Sons;
    Bhag S. Guru and Huseyin R. Hizirogulu, ``Electric Machinery and Transformers", Oxford University Press.
}