\begin{center}
    \textbf{\huge{\uppercase{Wireless Communications}}}
    \\
    \vspace{.5cm}
    \textbf{\large{EX 751}}
\end{center}

\noindent\textbf{Lecture\ \ \ : 3} \hfill \textbf{Year : IV} \\
\textbf{Tutorial \ : 0} \hfill \textbf{Part : II } \\
\textbf{Practical : 0}  \\

\par
\noindent 
\section*{Course Objectives:}
To introduce the student to the principles and building blocks of wireless communications.


\mylist{
    \textbf{Introduction \hfill (2 hours)}
    \mylist{
        Evolution of Wireless (mobile) communications, worldwide market, examples;
        Comparison of available wireless systems, trends;
        Trends in cellular radio (2G, 2.5G, 3G, beyond 3G) and personal wire wireless communication systems.
    };
    \textbf{Cellular mobile communication concept \hfill (4 hours)}
    \mylist{
        Frequency re-use and channel assignment strategies;
        Handoff strategies, types, priorities, practical considerations;
        Interference and system capacity, co-channel and adjacent channel interference, power control measures;
        Grade of Service, definition, standards;
        Coverage and capacity enhancement in cellular network, cell splitting, sectoring, repeaters, microcells
    };
    \textbf{Radio wave propagation in mobile network environment \hfill (12 hours)}
    \mylist{
        Review Free space propagation model, radiated power and electric field;
        Review Propagation mechanisms (large-scale path loss) -- Reflection, ground reflection, diffraction and scattering;
        Practical link budget design using path loss models;
        Outdoor propagation models (Longley-Rice, Okumura, Hata, Walfisch and Bertoni, microcell);
        Indoor propagation models (partition losses, long-distance path loss, multiple breakpoint, attenuation factor);
        Small scale fading and multipath (factors, Doppler shift), Impusle response model of multipath channel, multipath measurements, parameters of mobile multipath channel (time dispersion, coherence bandwidth, Doppler spread and coherence time);
        Types of small-scale fading (flat, frequency selective, fast, slow), Rayleigh and Ricean fading distribution
    };
    \textbf{Modulation-Demodulation methods in mobile communications \hfill (4 hours)}
    \mylist{
        Review of amplitude (DSB, SSB, VSB) and angle (frequency, phase) modulations and demodulation techniques;
        Review of line coding, digital linear (BPSK, DPSK, QPSKs) and constant envelop (BFSK, MSK, GMSK) modulation and demodulation techniques;
        M-ary (MPSK, MFSK, QAM and OFDM) modulation and demodulation techniques;
        Spread spectrum modulation techniques, PN sequences, direct sequence and frequency hopped spread spectrums;
        Performance comparison of modulations techniques in various fading channels
    };
    \textbf{Equalization and diversity techniques \hfill (4 hours)}
    \mylist{
        Basics of equalization. Equalization in communications receivers, linear equalizers;
        Non-linear equalization, decision feedback and maximum likelihood sequence estimation equalizations;
        Adaptive equalization algorithms, zero forcing, least mean square, recursive least square algorithms, fractionally spaced equalizers;
        Diversity methods, advantages of diversity, basic definitions;
        Space diversity, reception methods (selection, feedback, maximum ratio and equal gain diversity);
        Polarization, frequency and time diversity;
        RAKE receivers and interleaving
    };
    \textbf{Speech and channel coding fundamentals \hfill (4 hours)}
    \mylist{
        Characteristics of speech signal,s frequency domain coding of speech (sub-band and adaptive transform coding);
        Vocoders (channel, formant, cepstrum and voice-excited), Linear predictive coders (multipulse, code and residual excited LPCs), Codec for GSM mobile standard;
        Review of block codes, Hamming, Hadamard, Golay, Cyclic, Bosh-Chaudhary-Hocquenghgem(BCH), Reed-Solomon (RS) codes;
        Convolutional codes, encoders, coding gain, decoding algorithms (Viterbi and others);
        Trellis Code Modulation (TCM), Turbo codes
    };
    \textbf{Multiple Access in Wireless communications \hfill (9 hours)}
    \mylist{
        Frequency Division Multiple Access (FDMA), principles and applications;
        Time Division Multiple Access (TDMA), principles and applications;
        Spread Spectrum Multiple Access, Frequency Hopped Multiple Access, Code Division Multiple Access, Hybrid Spread Spectrum Multiple Access Techniques;
        Space Division Multiple Access;
        Standards for Wireless Local Area Networks
    };
    \textbf{Wireless systems and standards \hfill (6 hours)}
    \mylist{
        Evolution of wireless telephone systems: AMPS, PHS, DECT, CT2, IS-94, PACS, IS-95, IS-136, IS-54 etc.;
        Global system for Mobile (GSM): Services and features, system architecture, radio sub-system, channel types (traffic and control), frame structure, signal processing, example of a GSM call;
        CDMA standards: Frequency and channel specifications, Forward and Reverse CDMA channels;
        WiFi, WiMax, UMB, UMTS, CDMA-EVDO, LTE, and recent trends;
        Regulatory issues (spectrum allocation, spectrum pricing, licensing, tariff regulation and interconnection issues)
    }
}

\section*{Practicals:}
\mylist{
    Case study and field visit;
    Visit to mobile service operator, network service provider, internet service provider
}


\section*{References:}
\mylist{
    K. Feher, Wireless Digital Communications;
    T. Rappaport, Wireless Communications;
    J. Schiller, Mobile Communications;
    Leon Couch, Digital and analog communication systems;
    B.P. Lathi, Analog and Digital Communication systems;
    J. Proakis, Digital communication systems;
    D. Sharma, Course manual ``Communication Systems II"
}