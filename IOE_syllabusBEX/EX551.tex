\begin{center}
    \textbf{\huge{\uppercase{Microprocessors}}}
    \\
    \vspace{.5cm}
    \textbf{\large{EX 551}}
\end{center}

\noindent\textbf{Lecture\ \ \ : 3} \hfill \textbf{Year : II } \\
\textbf{Tutorial \ : 1} \hfill \textbf{Part : II } \\
\textbf{Practical : 3}\\

\par
\noindent 
\section*{Course Objective:}
To familiarize students with architecture, programming, hardware and application of microprocessor.


\mylist{
    \textbf{Introduction \hfill (4 hours)}
    \mylist{
        Introduction and History of Microprocessors;
        Basic Block Diagram of a Computer;
        Organization of Microprocessor Based System;
        Bus Organization;
        Stored Program Concept and Von Neumann Machine;
        Processing Cycle of a Stored Program Computer;
        Microinstructions and Hardwired/Microprogrammed Control Unit;
        Introduction to Register Transfer Language
    };
    \textbf{Programming with 8085 Microprocessor \hfill (10 hours)}
    \mylist{
        Internal Architecture and features of 8085 microprocessor;
        Instruction format and Data format;
        Addressing Modes of 8085;
        Intel 8085 Instruction set;
        Various Programs in 8085
        \mylist{
            Simple programs with arithmetic and logical operations;
            Conditions and loops;
            Array and Table processing;
            Decimal BCD Conversion;
            Multiplication and Division
        }
    };
    \textbf{Programming with 8086 Microprocessor \hfill (12 hours)}
    \mylist{
        Internal Architecture and Features of 8086 Microprocessor
        \mylist{
            BIU and Components;
            EU and Components;
            EU and BIU Operations;
            Segment and Offset Address
        };
        Addressing Modes of 8086;
        Assembly Language Programming;
        High level versus Low level Programming;
        Assembly Language Syntax
        \mylist{
            Comments;
            Reserved words;
            Identifiers;
            Statements;
            Directives;
            Operators;
            Instructions
        };
        EXE and COM programs;
        Assembling, Linking and Executing;
        One Pass and Two Pass Assemblers;
        Keyboard and Video Services;
        Various Programs in 8086
        \mylist{
            Simple Programs for Arithmetic, Logical, String Input/Output;
            Conditions and Loops;
            Array and String Processing;
            Read and Display ASCII and Decimal Numbers;
            Displaying Numbers in Binary and Hexadecimal Formats
        }
    };
    \textbf{Microprocessor System \hfill (10 hours)}
    \mylist{
        Pin Configuration of 8085 and 8086 Microprocessors;
        Bus Structure
        \mylist{
            Synchronous Bus;
            Asynchronous Bus;
            Read and Write Bus Timing of 8085 and 8086 Microprocessors
        };
        Memory Device Classification and Hierarchy;
        Interfacing I/O and Memory
        \mylist{
            Address Decoding;
            Unique and Non Unique Address Decoding;
            I/O Mapped I/O and Memory Mapped I/O;
            Serial and Parallel Interfaces;
            I/O Address Decoding with NAND and Block Decoders (8085, 8086);
            Memory Address Decoding with NAND, Block and PROM Decoders (8085, 8086)
        };
        Parallel Interface
        \mylist{
            Modes: Simple, Wait, Single Handshaking and Double Handshaking;
            Introduction to Programmable Peripheral Interface(PPI)
        };
        Serial Interface
        \mylist{
            Synchronous and Asynchronous Transmission;
            Serial Interface Standards: RS232, RS423, RS422, USB;
            Introduction to USART
        };
        Introduction to Direct Memory Access(DMA) and DMA controllers
    };
    \textbf{Interrupt Operations \hfill (5 hours)}
    \mylist{
        Polling versus Interrupt;
        Interrupt Processing Sequence;
        Interrupt Service Routine;
        Interrupt Processing in 8085
        \mylist{
            Interrupt Pins and Priorities;
            Using Programmable Interrupt Controllers (PIC);
            Interrupt Instructions
        };
        Interrupt Processing in 8086
        \mylist{
            Interrupt Pins;
            Interrupt Vector Table and its Organization;
            Software and Hardware Interrupts;
            Interrupt Priorities
        }
    };
    \textbf{Advanced Topics \hfill (4 hours)}
    \mylist{
        Multiprocessing Systems
        \mylist{
            Real and Pseudo-Parallelism;
            Flynn's Classification;
            Instruction Level, Thread Level and Process Level Parallelism;
            Interprocess Communication, Resource Allocation and Deadlock;
            Features of Typical Operating System
        };
        Different Microprocessor and Architectures
        \mylist{
            Register Based and Accumulator Based Architecture;
            RISC and CISC Architecture;
            Digital Signal Processors
        }
    }
}

\section*{Practical:}
There will be about 12 lab exercises to program 8085 and 8086 microprocessors


\section*{References:}
\mylist{
    Ramesh S. Gaonkar, ``Microprocessor Architecture, Programming and Application with 8085", Prentice Hall;
    Peter Abel, ``IBM PC Assembly Language and Programming", Pearson Education Inc.;
    D.V. Hall, ``Microprocessor and Interfacing, Programming and Hardware", Tata McGraw Hill.;
    John Uffenbeck, ``Microcomputers and Microprocessors, The 8080, 8085 and Z-80 Programming, Interfacing and Troubleshooting", Prentice Hall.;
    Water A. Triebel and Avatar Singh, ``The 8088 and 8086 Microprocessors, Programming, Interfacing, Software, Hardware and Applications", Prentice Hall.;
    William Stalling,``Computer Orgarnization and Architecture", Prentice Hall.
}