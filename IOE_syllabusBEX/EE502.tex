\begin{center}
    \textbf{\huge{\uppercase{Electrical Engineering Material}}}
    \\
    \vspace{.5cm}
    \textbf{\large{EE 502}}
\end{center}

\noindent\textbf{Lecture\ \ \ : 3} \hfill \textbf{Year : II } \\
\textbf{Tutorial \ : 1} \hfill \textbf{Part : I } \\
\textbf{Practical : 0}  \\

\par
\noindent 
\section*{Course Objective:}
To provide a basic understanding of the different materials used in electrical and electronics engineering.


\mylist{
    \textbf{Theory of Metals \hfill (8 hours)}
    \mylist{
        Elementary quantum mechanical idea: wave particle duality, wave function, Schrodinger's equation, operator notation, expected value;
        Infinite potential well: A confined electron;
        Finite potential barrier: Tunneling phenomenon;
        Free electron theory of metals: electron in a linear solid, Fermi energy, Degenerate states, Number of states, Density of state, Population density;
        Fermi-Dirac Distribution function;
        Thermionic emission: Richardson's equation, Schottky effect;
        Contact potential: Fermi level at equilibrium.
    };
    \textbf{Free electron theory of conduction in metal \hfill (6 hours)}
    \mylist{
        Crystalline structure: Simple cubic structure, body centered cubic, face centered cubic;
        Band theory of solids;
        Effective mass of electron;
        Thermal velocity of electron at equilibrium;
        Electron mobility, conductivity and resistivity
    };
    \textbf{Dielectric materials \hfill (6 hours)}
    \mylist{
        Matter polarization and relative permittivity: Relative permittivity, Dipole moment, polarization vector, local field, Clausius-Mossotti equation;
        Types of Polarization: electronic polarization, ionic polarization, orientational polarization, interfacial polarization;
        Dielectric losses: frequency dependence;
        Dielectric breakdown in solids;
        Ferro-electricity and Piezo-electricity
    };
    \textbf{Magnetic materials \hfill (6 hours)}
    \mylist{
        Magnetic material classification: diamagnetismm, paramagnetism, ferromagnetism, Anti-ferromagnetism, ferrimagnetism;
        Magnetic domain: Domain structure, domain wall motion, Hysteresis loop, Eddy current losses, demagnetization;
        Soft magnetic materials: Examples and uses;
        Hard magnetic materials: Examples and uses
    };
    \textbf{Superconductivity \hfill (5 hours)}
    \mylist{
        Zero Resistance and the Meissner effect;
        Type I and Type II superconductors;
        Critical current density
    };
    \textbf{Semiconductors \hfill (14 hours)}
    \mylist{
        Intrinsic semiconductors: Silicon crystal, energy band diagram, conduction in semiconductors, electrons and hole concentration.;
        Extrinsic semiconductors: n-type doping, p-type doping, compensation doping;
        Introduction to GaAs semiconductor;
        Temperature dependence of conductivity: Carrier concentration temperature dependence, drift mobility temperature and impurity dependence, conductivity temperature dependence, degenerate and non-degenerate semiconductors;
        Diffusion on semiconductor: Einstein relationship;
        Direct and indirect generation and recombination;
        PN junction: Forward biased, reverse biased PN-junction
    }
}

\section*{References:}
\mylist{
    Bhadra Prasas Pokharel and Nava Raj Karki, ``Electrical Engineering Materials", Sigma Offset press, Kamaladi, Kathmandu, Nepal;
    R.C. Jaeger, ``Introduction to Microelectronic Fabrication- Volume IV", Addison Wesley publishing Company, Inc.;
    Kasam S.O, ``Principles of electrical engineering materials and devices, McGraw Hill, New York.;
    R.A. Colcaser and S. Diehl-Nagle, ``Materials and Devices for Electrical Engineers and Physicists", McGraw Hill, New York.
}