\begin{center}
    \textbf{\huge{\uppercase{Engineering Drawing I}}}
    \\
    \vspace{.5cm}
    \textbf{\large{ME 401}}
\end{center}

\noindent\textbf{Lecture\ \ \ : 1} \hfill \textbf{Year : I } \\
\textbf{Tutorial \ : 0} \hfill \textbf{Part : I } \\
\textbf{Practical : 3}  \\

\par
\noindent 
\section*{Course Objective:}
To develop basic projection concepts with reference to points, lines, planes and geometrical solids. Also to develop sketching and drafting skills to facilitate communication.

\begin{enumerate}
    \item \textbf{Instrumental Drawing, Technical Lettering Practices \& Techniques \hfill (2 hours) }
    \begin{enumerate}
        \item Equipment and materials
        \item Description of drawing instruments, auxiliary equipment and drawing materials
        \item Techniques of instrumental drawing 
        \item Pencil sharpening, securing paper, proper use of T-squares, triangles scales dividers, compasses, erasing shields, French curves, inking pens
        \item Lettering strokes, letter proportions, use of pencils and pens, uniformity and appearance of letters, freehand techniques, inclined and vertical letters and numerals, upper and lower cases, standard English lettering forms
    \end{enumerate}
    
    \item \textbf{Dimensioning \hfill (2 hours)}
    \begin{enumerate}
        \item Fundamentals and techniques
        \item Size and location dimensioning, SI conversions
        \item Use of scales, measurement units, reducing and enlarging drawings
        \item Placement of dimensions: aligned and unidirectional
    \end{enumerate}
    
    \item \textbf{Applied Geometry \hfill (6 hours)}
    \begin{enumerate}
        \item Plane geometrical construction: Proportional division of lines, arc \& line tangents
        \item Methods for drawing standard curves such as ellipses, parabolas, hyperbolas, involutes, spirals, cycloids and helices (cylindrical and conical)
        \item Techniques to reproduce a given drawing (by construction)
    \end{enumerate}
    
    \item \textbf{Basic Descriptive Geometry \hfill (14 hours)}
    \begin{enumerate}
        \item Introduction to Orthographic projection, Principal Planes, Fours Quadrants or Angles
        \item Projection of points on first, second, third and fourth quadrants
        \item Projection of Lines: Parallel to one of the principal plane, inclined to one of the principal plane and parallel to other, inclined to both principal planes
        \item Projection Planes: Perpendicular to both principal planes, Parallel to one of the principal planes and Inclined to one of the principal planes, perpendicular to other and inclined to both principal planes
        \item True length of lines: horizontal, inclined and oblique lines
        \item Rules for parallel and perpendicular lines
        \item Point view and end view of a line
        \item Shortest distance from a point to a line
        \item Edge View and True shape of an oblique plane
        \item Angle between two intersecting lines
        \item Intersection of a line and a plane
        \item Angle between two intersecting lines
        \item Dihedral angle between two planes
        \item Shortest distance between two skew lines
        \item Angel between two non-intersecting(skew) lines
    \end{enumerate}
    
    
    \item \textbf{Multi view (orthographic) projections \hfill (18 hours)}
    \begin{enumerate}
        \item Orthographic Projections
        \begin{enumerate}
            \item First and third angle projection
            \item Principal views: methods for obtaining orthographic views, Projection of lines, angles and plane surfaces, analysis in three views, projection of curved lines and surfaces, object orientation and selection of views for best representation, full and hidden lines
            \item Orthographic drawings: making an orthographic drawing, visualizing objects (pictorial view) from the given views
            \item Interpretation of adjacent areas, true-length lines, representation of holes, conventional practices
        \end{enumerate}
        
        \item Sectional Views: Full, half, broken revolved, removed(detail) sections, phantom of hidden section, Auxiliary sectional views, specifying cutting planes for sections, conventions for hidden lines, holes, ribs, spokes
        
        \item Auxiliary views: Basic concept and use, drawing methods and types, symmetrical and unilateral auxiliary views. Projection of curved lines and boundaries, line of intersection between two planes, true size of dihedral angles, true size and shape of plane surfaces.
    \end{enumerate}
    
    \item \textbf{Developments and Intersections \hfill (18 hours)}
    \begin{enumerate}
        \item Introduction and Projection of Solids
        \item Developments: general concepts and practical considerations, development of a right or oblique prism, cylinder, pyramid, and cone, development of truncated pyramid and cone, Triangulation method for approximately developed surfaces, transition pieces for connecting different shapes, development of a sphere.
        \item Intersections: lines of intersection of geometric surfaces, piercing point of a line and a geometric solid, intersection lines of two planes, intersections of prisms and pyramids, cylinder and an oblique plane. Constructing a development using auxiliary views, intersection of two cylinders, a cylinder \& a cone.
    \end{enumerate}
\end{enumerate}

\section*{Practical}
\begin{enumerate}
    \item Drawing Sheet Layout, Freehand Lettering, Sketching of parallel lines, circles, Dimensioning
    \item Applied Geometry (Sketch and Instrumental Drawing)
    \item Descriptive Geometry I: Projection of point and lines (4.1 to 4.3) (Sketch and Instrumental Drawing)
    \item Descriptive Geometry II: Projection of Planes (4.4) (Sketch and Instrumental Drawing)
    \item Descriptive Geometry III: Applications in Three dimensional space (4.5 to 4.15) (Sketch and Instrumental Drawing)
    \item Multiview Drawings (5.1) (Sketch and Instrumental Drawing)
    \item Multiview, Sectional Drawings and Dimensioning I (5.2) (Sketch and Instrumental Drawing)
    \item Multiview, Sectional Drawings and Dimensioning II (5.2) (Sketch and Instrumental Drawing)
    \item Multiview, Sectional Drawings and Dimensioning III (5.3) (Sketch and Instrumental Drawing)
    \item Projection of Regular Geometrical Solids (Sketch and Instrumental Drawing)
    \item Development of Intersection I (6.1) (Sketch and Instrumental Drawing)
    \item Development of Intersection II (6.2) (Sketch and Instrumental Drawing)
    \item Development of Intersection III (6.3) (Sketch and Instrumental Drawing)
\end{enumerate}


\section*{References:}
\begin{enumerate}
    \item M. C. Luintel, ``Engineering Drawing (Vol.I)", Athrai Publication (P) Limited.
    \item W. J. Luzadder, ``Fundamentals of Engineering Drawing", Prentice Hall.
    \item T. E. French, C. J. Vierck, and R. J. Foster, ``Engineering Drawing and Graphic Technology", Mc Graw Hill Publishing Co.
    \item A. Mitchell, H.C. Spencer and J. T. Dygdone, ``Technical Drawing", F. E. Giescke, Macmillan Publishing Co.
    \item N. D. Bhatt, ``Elementary Engineering Drawing", Charotar Publishing House, India.
    \item P. S. Gill, ``A Text Book of Engineering Drawing", S.K. Kataria and Sons, India.
    \item R. K. Dhawan, ``A Text Book of Engineering Drawing", S. Chand and Company Limited, India.
\end{enumerate}