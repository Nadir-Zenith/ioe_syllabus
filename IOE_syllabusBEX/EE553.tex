\begin{center}
    \textbf{\huge{\uppercase{Power System}}}
    \\
    \vspace{.5cm}
    \textbf{\large{EE 553}}
\end{center}

\noindent\textbf{Lecture\ \ \ : 3} \hfill \textbf{Year : II } \\
\textbf{Tutorial \ : 1} \hfill \textbf{Part : II } \\
\textbf{Practical : 0}  \\

\par
\noindent 
\section*{Course Objective:}
To deliver the principle and fundamental analysis techniques for generation, transmission and distribution components of a power system with basic protection system.


\mylist{
    \textbf{General Background \hfill (4 hours)}
    \mylist{
        Power System Evolution;
        Generation, Transmission, and Distribution components;
        Major electrical components in power station{;} alternators, transformers, bus bar, voltage regulators, switch and isolators, metering and control panels;
        Voltage levels, AC vs DC Transmission;
        Single phase and three phase power delivery;
        Single line diagram representation of a power system
    };
    \textbf{Mechanical consideration of Transmission \hfill (8 hours)}
    \mylist{
        Overhead lines
        \mylist{
            Line supports, spacing between conductors;
            Calculation of sag, equal and unequal supports, effect of ice and wind loadings;
            Application of GPS system
        };
        Underground cables
        \mylist{
            Classification, construction of cables, insulation resistance;
            Dielectric stress in single core/multi core cables;
            Cable faults and location of faults
        }
    };
    \textbf{Line parameter calculations \hfill (10 hours)}
    \mylist{
        Inductance, resistance and capacitance of a line;
        Inductance of line due to internal and external flux linkage;
        Skin and proximity effect;
        Inductance of single phase two wire line, stranded and bundled conductor consideration, concept of GMR and GMD, inductance of 3 phase line{;} equilateral and unsymmetrical spacing;
        Transposition, inductance of double circuit 3 phase lines;
        Concept of GMR and GMD for capacitance calculations;
        Capacitance calculations of single phase two wire line, stranded and bundled conductor consideration, capacitance of 3 phase lines, equilateral and unsymmetrical spacing, double circuit;
        Earth effect in capacitance of a line
    };
    \textbf{Transmission line performance analysis \hfill (8 hours)}
    \mylist{
        Classification of a line based on short, medium and long lines;
        Representation of `Tee' and `Pi' of medium lines{;} calculation of ABCD parameters;
        Per unit system{;} advantage and applications;
        Voltage regulations and efficiency calculation of transmission lines;
        Transmission line as source and sink of reactive power;
        Real and reactive power flow through lines;
        Surge impedance loading;
        Reactive compensation of transmission lines
    };
    \textbf{Interconnected power system \hfill (5 hours)}
    \mylist{
        Real power/ frequency balance;
        Reactive power/ voltage balance;
        Computer application in interconnected power system;
        Basic concept of Power system load flow
    };
    \textbf{Distribution System \hfill (5 hours)}
    \mylist{
        Distribution system terminology;
        Distribution transformer and load centers;
        Rural vs urban distribution;
        Radial, loop, and network distribution;
        Voltage drop computation in a radial DC and AC distribution
    };
    \textbf{Introduction to power system protection \hfill (5 hours)}
    \mylist{
        Power system faults and protection principle;
        Fuse as a protection device;
        Relays{;} working and types;
        Circuit breaker{;} working and types;
        Basic protection schemes for generators, motors, transformers and transmission lines;
        Basic concept of power line carrier communication (PLCC)
    }
}

\section*{References:}
\mylist{
    W,D, Stevension, ``Power System Analysis", Tata McGraw Hill Publications;
    P.N. Singh, ``Electric power Generation, Transmission \& Distribution", Prentice Hall.
}