\begin{center}
    \textbf{\huge{\uppercase{Electric Circuit Theory}}}
    \\
    \vspace{.5cm}
    \textbf{\large{EE 501}}
\end{center}

\noindent\textbf{Lecture\ \ \ : 3} \hfill \textbf{Year : II } \\
\textbf{Tutorial \ : 1} \hfill \textbf{Part : I } \\
\textbf{Practical : 3/2}  \\

\par
\noindent 
\section*{Course Objective:}
To continue work in Basic Electrical Engineering including the use of Laplace Transform to determine the time and frequency domain responses of electrical circuits.

\mylist{
    \textbf{Network Analysis of AC circuit \& dependent sources \hfill (8 hours)}
    \mylist{
        Mesh Analysis;
        Nodal Analysis;
        Series \& parallel resonance in RLC circuits 
        \mylist{
            Impedance and phase angle of series Resonant Circuit;
            Voltage and current in series resonant circuit;
            Band width of the RLC circuit;
            High-Q and Low-Q circuits
        }
    };
    \textbf{Initial Conditions \hfill (2 hours)}
    \mylist{
        Characteristics of various network elements;
        Initial value of derivatives;
        Procedure for evaluating initial conditions;
        Initial condition in the case of R-L-C network
    };
    \textbf{Transient analysis in RLC circuit by direct solution \hfill (10 hours)}
    \mylist{
        Introduction;
        First order differential equation;
        Higher order homogeneous and non-homogeneous differential equations;
        Particular integral by method of undetermined coefficients;
        Response of R-L circuit with
        \mylist{
            DC excitation;
            Exponential excitation;
            Sinusoidal excitation
        };
        Response of R-C circuit with
        \mylist{
            DC excitation;
            Exponential excitation;
            Sinusoidal excitation
        };
        Response of series R-L-C circuit with
        \mylist{
            DC excitation;
            Exponential excitation;
            Sinusoidal excitation
        };
        Response of parallel R-L-C circuit with DC excitation
    };
    \textbf{Transient analysis of RLC circuit by Laplace Transform \hfill (8 hours)}
    \mylist{
        Introduction;
        The Laplace Transformation;
        Important properties of Laplace transformation;
        Use of Partial Fraction expansion in analysis using Laplace Transformations;
        Heaviside's partial fraction expansion theorem;
        Response of R-L circuit with
        \mylist{
            DC excitation;
            Exponential excitation;
            Sinusoidal excitation
        };
        Response of R-C circuit with
        \mylist{
            DC excitation;
            Exponential excitation;
            Sinusoidal excitation
        };
        Response of series R-L-C circuit with
        \mylist{
            DC excitation;
            Exponential excitation;
            Sinusoidal excitation
        };
        Response of parallel R-L-C circuit with exponential excitation;
        Transfer functions Poles and Zeros of Networks
    };
    \textbf{Frequency Response of Network \hfill (6 hours)}
    \mylist{
            Introduction;
            Magnitude and Phase response;
            Bode diagrams;
            Band width of Series and parallel resonance circuits;
            Basic concept of filters, high pass, low pass, band pass and band stop filters
    };
    \textbf{Fourier Series and transform \hfill (5 hours)}
    \mylist{
        Basic concept of Fourier series and analysis;
        Evaluation of Fouier coefficients for periodic non-sinusoidal waveforms in electric networks;
        Introduction of Fourier transforms
    };
    \textbf{Two-port Parameter of Networks \hfill (6 hours)}
    \mylist{
        Definition of two-port networks;
        Short circuit admittance parameters;
        Open circuits impedance parameters;
        Transmission Short circuit admittance parameters;
        Hybrid parameters;
        Relationship and transformations between sets of parameters;
        Application to filters;
        Applications to transmission lines;
        Interconnection of two-port network (Cascade, series, parallel)
    }
}


\section*{Practical:}
\mylist{
    Resonance in RLC series circuit \\ - measurement of resonant frequency;
    Transient Response in first order system passive circuits \\ - measure step and impulse response of RL and RC circuits using oscilloscope \\ - relate time response to analytical transfer functions calculations;
    Transient Response in Second Order system passive circuits \\ -  measure step and impulse response of RLC series and parallel circuits using oscilloscope \\ - relate time response to transfer functions and pole-zero configuration;
    Frequency Response of first order passive circuits \\ - measure amplitude and phase response and plot bode diagrams for RL, RC and RLC circuits \\ - relate Bode diagrams to transfer functions and pole zero configuration circuit;
    Frequency Response of second order passive circuits \\ - measure amplitude and phase response and plot bode diagrams for RL, RC and RLC circuits \\ - relate Bode diagrams to transfer functions and pole zero configuration circuit
}

\section*{References:}
\mylist{
    M.E. Van Valkenburg, ``Network Analysis", Prentice Hall;
    William H. Hyat Jr. \& Jack E. Kemmerly, ``Engineering Circuits Analysis", McGraw Hill International Editions, Electrical Engineering series;
    Michel D. Cilletti, ``Introduction to Circuit Analysis and Design", Holt, Hot Rinehart and Winston International Edition, New York
}