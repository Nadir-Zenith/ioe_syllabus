\begin{center}
    \textbf{\huge{\uppercase{Applied Mathematics}}}
    \\
    \vspace{.5cm}
    \textbf{\large{SH 551}}
\end{center}

\noindent\textbf{Lecture\ \ \ : 3} \hfill \textbf{Year : II } \\
\textbf{Tutorial \ : 1} \hfill \textbf{Part : II } \\
\textbf{Practical : 0}  \\

\par
\noindent 
\section*{Course Objective:}
This course focuses on several branches of applied mathematics. The students are exposed to complex variable theory and a study of the Fourier and Z-transforms, topics of current importance in signal processing. The course concludes with studies of the wave and heat equations in Cartesian and polar coordinates.


\mylist{
    \textbf{Complex Analysis \hfill (18 hours)}
    \mylist{
        Complex Analytic Functions
        \mylist{
            Functions and sets in the complex plane;
            Limits and Derivatives of complex functions;
            Analytic functions. The Cauchy-Riemann equations;
            Harmonic functions and it's conjugate
        };
        Conformal Mapping
        \mylist{
            Mapping;
            Some familiar functions as mappings;
            Conformal mappings and special linear functional transformations;
            Constructing conformal mappings between given domains
        };
        Integral in the Complex Plane
        \mylist{
            Line integrals in the complex plane;
            Basic Problems of the complex line integrals;
            Cauchy's integral theorem;
            Cauchy's integral formula;
            Supplementary problems
        };
        Complex Power Series, Complex Taylor series and Lauren series
        \mylist{
            Complex power series;
            Functions represented by power series;
            Taylor series, Taylor series of elementary functions;
            Practical methods for obtaining power series, Lauren series;
            Analyticity at infinity, zeros, singularities, residues, Cauchy's residue theorem;
            Evaluation of real integrals
        };
    };
    \textbf{The Z-Transform \hfill (9 hours)}
    \mylist{
        Introduction;
        Properties of Z-Transform;
        Z-transform of elementary functions;
        Linearity properties;
        First shifting theorem, second shifting theorem, Initial valuer theorem;
        Final value theorem, Convolution theorem;
        Some standard Z-transform;
        Inverse Z-transform;
        Method for finding Inverse Z-transform;
        Application of Z-transform to difference equations
    };
    \textbf{Partial Differential Equations \hfill (12 hours)}
    \mylist{
        Linear partial differential equation of second order, their classification and solution;
        Solution of one dimensional wave equation, one dimensional heat equation, two dimensional heat equation and Laplace equation (Cartesian and polar form) by variable separation method.
    };
    \textbf{Fourier Transform \hfill (6 hours)}
    \mylist{
        Fourier integral theorem, Fourier sine and cosine integral{;} Complex form of Fourier integral;
        Fourier transform, Fourier sine transform, Fourier cosine transform and their properties;
        Convolution, Parseval's identity for Fourier transforms;
        Relation between Fourier transform and Laplace transforms
    }
}

\section*{References:}
\mylist{
    S.K. Mishra, G.B. Joshi, S. Ghimire, V. Parajuli, ``A textbook of  Applied Mathematics", Dibya Deurali Prakashan;
    E. Kreyzig, ``Advance Engineering Mathematics", Fifth Edition, Wiley, New York;
    A.V. Oppenheim, ``Discrete-Time Signal Processing", Prentice Hall;
    K. Ogata, ``Discrete-Time Control System", Prentice Hall, Englewood Cliffs, New Jersey, 1987
}