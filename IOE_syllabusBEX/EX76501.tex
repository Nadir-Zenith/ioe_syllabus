\begin{center}
    \textbf{\huge{\uppercase{Optical Fiber Communication \\ \vspace{0.5cm} system }}}
    \\
    \vspace{.5cm}
    \textbf{\large{EX 765 01}}
\end{center}

\noindent\textbf{Lecture\ \ \ : 3} \hfill \textbf{Year : IV} \\
\textbf{Tutorial \ : 1} \hfill \textbf{Part : II } \\
\textbf{Practical : 3/2}  \\

\par
\noindent 
\section*{Course Objectives:}
To introduce the concept of optical fiber communication.

\mylist{
    \textbf{Introduction to Optical Fiber Communication \hfill (2 hours)}
    \mylist{
        Evolution of optical fiber communication;
        Optical fiber communication system;
        Advantages of optical fiber communication;
        Applications of optical fiber communication
    };
    \textbf{Light Transmission in Optical Fiber \hfill (2 hours)}
    \mylist{
        Introduction of optical fiber structure;
        Total internal reflection;
        Acceptance angle;
        Numerical aperture;
        Meridional and skew rays in optical wave guide
    };
    \textbf{Electromagnetic Theory for Optical Propagation \hfill (2 hours)}
    \mylist{
        Review of Maxwell's equation;
        The wave equation for slab waveguide;
        Wave equation for cylindrical waveguide
    };
    \textbf{Mode Propagation in Optical Waveguide \hfill (3 hours)}
    \mylist{
        Modes in a planar optical guide;
        Phase and group velocity;
        Evanescent field;
        Modes in cylindrical optical waveguide;
        Mode coupling
    };
    \textbf{Optical Fibers \hfill (5 hours)}
    \mylist{
        Introduction and types;
        Modes in multimode fibers: step index and graded index;
        Modes in step index and graded index single mode fiber;
        Cutoff wavelength, mode-field diameter and spot size;
        Transmission properties of optical fiber;
        Fiber attenuation;
        Fiber bend loss;
        Fiber dispersion
    };
    \textbf{Optical Source for Optical Fiber Communication \hfill (4 hours)}
    \mylist{
        Introduction, types and requirements;
        Light emitting diode (LED);
        Laser diode (LD);
        Properties of optical sources
    };
    \textbf{Optical Detectors \hfill (4 hours)}
    \mylist{
        Introduction;
        Semiconductor photodiode;
        PIN photodiode;
        Avalanche photodiode;
        Comparision of different photodiodes;
        Properties of photodiodes
    };
    \textbf{Optical Modulation \hfill (3 hours)}
    \mylist{
        Introduction and types;
        Analog modulation;
        Digital modulation
    };
    \textbf{Connectors and Couplers \hfill (6 hours)}
    \mylist{
        Introduction to optical connections;
        Optical fiber connectors: Principle and types;
        Characteristic losses in connectors;
        Optical fiber splices: Principle and types;
        Comparison of different types of splices;
        Comparison between splice and connector;
        Introduction to optical couplers and their types;
        Fused biconical taper (bus) coupler;
        Fused star coupler;
        Characteristic properties of optical couplers;
        Fully bidirectional four port optical coupler;
        Asymmetrical bidirectional three port optical coupler (ABC);
        Comparison between four port full bidirectional coupler made with traditional three port coupler and ABC
    };
    \textbf{Fiber Amplifiers and Integrated Optics \hfill (4 hours)}
    \mylist{
        Introduction;
        Rare earth doped fiber amplifier;
        Raman and Brillouin fiber amplifier;
        Integrated optics;
        Optical Switch
    };
    \textbf{Optical Fiber Network \hfill (10 hours)}
    \mylist{
        Introduction to analog and digital fiber optic transmission;
        Optical fiber local area networks;
        Design of passive digital fiber optic networks
    }
}


\section*{Practicals:}
\mylist{
    Familiarization with optical fiber laboratory, safety and precaution;
    Demonstration of the concept of light propagation in optical waveguide with the help of polymer rod and water spout;
    Determination of fiber numerical aperture and fiber attenuation;
    Plotting a power-current characteristic for LED;
    Determination of different optical fiber connector losses.
    Determination of coupling efficiency/loss from source to fiber, fiber to fiber, and fiber to photodetector;
    Digital optical transmission
}


\section*{References:}
\mylist{
    John M. Senior, ``Optical Fiber Communications -- Principles and Practice", Prentice Hall;
    William B. Jones Jr., ``Introduction to Optical Fiber Communication Systems", Holt, Rinheart and Winston,Inc.;
    Gerd Keiser, ``Optical Fiber Communication", Second Edition, McGraw Hill Inc.;
    Roshan Raj Karmacharya, ``Passive Optical Fiber LAN Design", M.Sc. Thesis, University of Calgary, Canada.
}