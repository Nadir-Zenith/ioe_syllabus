\begin{center}
    \textbf{\huge{\uppercase{Digital Signal Processing}}}
    \\
    \vspace{.5cm}
    \textbf{\large{EX 753}}
\end{center}

\noindent\textbf{Lecture\ \ \ : 3} \hfill \textbf{Year : IV} \\
\textbf{Tutorial \ : 1} \hfill \textbf{Part : II } \\
\textbf{Practical : 3/2}  \\

\par
\noindent 
\section*{Course Objectives:}
To introduce digital signal processing techniques and applications, and to design and implement IIR and FIR digital filter.


\mylist{
    \textbf{Introduction \hfill (4 hours)}
    \mylist{
        Basic elements of Digital Signal Processing;
        Need of Digital Signal Processing over Analog Signal Processing;
        A/D and D/A conversion;
        Sampling continuous signals and spectral properties of sampled signals
    };
    \textbf{Discrete-time Signals and System \hfill (6 hours)}
    \mylist{
        Elementary discrete-time signals;
        Linearity, Shift invariance, Causality of discrete systems;
        Recursive and Non-recursive discrete-time systems;
        Convolution sum and impulse response;
        Linear Time-invariant systems characterized by constant coefficient difference equations;
        Stability of LTI systems, Implementation of LTI system.
    };
    \textbf{Z-Transform \hfill (6 hours)}
    \mylist{
        Definition of the z-transform;
        One-side and two-side transforms, ROC, Left-side, Right-sided and two-sided sequences, Region of convergence, Relationship to causality;
        Inverse z-transform-by long division, by partial fraction expansion;
        Z-transform properties-- dealay advance, Convolution, Parseval's theorem;
        Z-transform function H(z)-transient and steady state sinusoidal response, pole-zero relationship stability
    };
    \textbf{Discrete Fourier Transform \hfill (7 hours)}
    \mylist{
        Definition and applications, Frequency domain sampling and for reconstruction, Forward and Reverse transforms, Relationship of the DFT to other transforms;
        Properties of Discrete Fourier Transform: Periodicity, Linearity, and Symmetry Properties, Multiplication of two DFTs and Circular Convolution, Time reversal, Circular time shift and Multiplication of two sequences circular frequency shift, Circular correlation and Parseval's Theorem;
        Efficient computation of the DFT: Algorithms, applications, applications of FFT algorithms
    };
    \textbf{Implementation of Discrete-time System \hfill (8 hours)}
    \mylist{
        Structures of FIR and IIR, Direct Form, Cascaded and parallel form, Lattice for FIR;
        Conversion between direct form and lattice and vice versa, Lattice and lattice-ladder for IIR;
        Frequency response;
        Digital filters, finite precision implementations of discrete filters;
        Representation of Numbers: Fixed point and floating binary point, Effect of Rounding and truncation, Limit cycle oscillations effect;
        Quantization of filter coefficients and effects on location of poles, and zeros, poles perturbation, Overflow and underflow error, Scaling to prevent overflow and underflow
    };
    \textbf{IIR Filter Design \hfill (5 hours)}
    \mylist{
        IIR Filter Design: IIR Filter design by classical filter design using low pass approximations Butterworth, Chebychev, Inverse Chebyshev, Elliptic and Bessel-Thompson filters;
        IIR filter design by Impulse-invariant method, Bilinear Transformation Method, Mathched z-transform method;
        IIR Lowpass discrete filter design using bilinear transformation;
        Spectral transformations, highpass, bandpass and Notch filters
    };
    \textbf{FIR Filter Design \hfill (5 hours)}
    \mylist{
        FIR filter design by Fourier approximation;
        Gibbs phenomena in FIR filter design, Design of Linear Phase FIR filters using window function, Applications of window functions to frequency response smoothing;
        Window functions, Rectangular, Hamming, Blackman, and Kaiser windows;
        Design of linear phase FIR filter by the frequency sampling method;
        FIR filter design using the Remez exchange algorithm;
        Design of optimum equiripple linear-phase FIR filters
    };
    \textbf{Digital Filter Implementation \hfill (4 hours)}
    \mylist{
        Implementations using special purpose DSP processor;
        Bit-serial arithmetic, pipelined implementations;
        Distributed arithmetic implementations.
    }
}

\section*{Practical:}
\mylist{
    Study the behavior of a simple digital notch filter;
    Response of a recursive digital filter;
    Scaling, dynamic range and noise behaviour of a recursive digital filter, observation of nonlinear finite precision effects;
    Response of a non-recursive digital filter, Implementation in Impulsive Invariant and Bilinear Transformation;
    Band pass filters implemented using cascade second order sections and wave or ladder filters, Comparison of implementations;
    Design of FIR filter using window method, Comparison of FIR filter for different windowing method.
}


\section*{References:}
\mylist{
    J.G. Proakis and D.G. Manolakis, ``Digital Signal Processing", Prentice Hall;
    A.V. Oppenheim, `` Discrete-Time Signal Processing", Prentice Hall;
    S.K. Mitra, ``Digital Signal Processing, A computer- based Approach", McGraw Hill
}