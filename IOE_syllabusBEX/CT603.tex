\begin{center}
    \textbf{\huge{\uppercase{Computer Organization and Architecture}}}
    \\
    \vspace{.5cm}
    \textbf{\large{CT 603}}
\end{center}

\noindent\textbf{Lecture\ \ \ : 3} \hfill \textbf{Year : III} \\
\textbf{Tutorial \ : 1} \hfill \textbf{Part : I } \\
\textbf{Practical : 3/2}  \\

\par
\noindent 
\section*{Course Objective:}
To provide the organization, architecture and designing concept of computer system including processor architecture, computer arithmetic, memory system, bus organization and multiprocessors.


\mylist{
    \textbf{Introduction \hfill (3 hours)}
    \mylist{
        Computer organization and architecture;
        Structure and function;
        Designing for performance;
        Computer components;
        Computer function;
        Interconnection structures;
        Bus Interconnection;
        PCI
    };
    \textbf{Central processing Unit \hfill (10 hours)}
    \mylist{
        CPU structure and function;
        Arithmetic and logic Unit;
        Instruction formats;
        Addressing modes;
        Data transfer and manipulation;
        RISC and CISC;
        64-bit Processor
    };
    \textbf{Control Unit \hfill (6 hours)}
    \mylist{
        Control Memory;
        Addressing sequencing;
        Computer Configuration;
        Micro-instruction format;
        Symbolic Micro-instructions;
        Symbolic Micro-program;
        Control Unit Operation;
        Design of control unit
    };
    \textbf{Pipeline and Vector processing \hfill (5 hours)}
    \mylist{
        Pipelining;
        Parallel processing;
        Arithmetic Pipeline;
        Instruction Pipeline;
        RISC pipeline;
        Vector processing;
        Array processing
    };
    \textbf{Computer Arithmetic \hfill (8 hours)}
    \mylist{
        Addition algorithm;
        Subtraction algorithm;
        Multiplication algorithm;
        Division algorithms;
        Logical operation
    };
    \textbf{Memory System \hfill (5 hours)}
    \mylist{
        Microcomputer Memory;
        Characteristics of memory systems;
        The Memory Hierarchy;
        Internal and External memory;
        Cache memory principles;
        Elements of Cache design
        \mylist{
            Cache size;
            Mapping function;
            Replacement algorithm;
            Write policy;
            Number of caches
        }
    };
    \textbf{Input-Output organization \hfill (6 hours)}
    \mylist{
        Peripheral devices;
        I/O modules;
        Input-output interface;
        Modes of transfer
        \mylist{
            Programmed I/O;
            Interrupt-driven I/O;
            Direct Memory Access
        };
        I/O processor;
        Data Communication processor
    };
    \textbf{Multiprocessors \hfill (2 hours)}
    \mylist{
        Characteristics of multiprocessors;
        Interconnection Structures;
        Interprocessor Communication and Synchronization
    }
}

\section*{Practical:}
\mylist{
    Addition of two unsigned Integer binary number;
    Multiplication of two unsigned integer binary numbers by partial-product method;
    Subtraction of two unsigned integer binary number;
    Division using Restoring;
    Division using non-restoring methods;
    To simulate a direct mapping cache
}

\section*{References:}
\mylist{
    M. Morris Mano, ``Computer System Architecture";
    William Stalling, ``Computer organization and architecture";
    John P. Hayes, ``Computer Architecture and Organization";
    V.P. Heuring, H.F. Jordan, ``Computer System design and architecture";
    S. Shakya, ``Lab Manual on Computer Architecture and design"
}