\begin{center}
    \textbf{\huge{\uppercase{RF and Microwave Engineering}}}
    \\
    \vspace{.5cm}
    \textbf{\large{EX 752}}
\end{center}

\noindent\textbf{Lecture\ \ \ : 3} \hfill \textbf{Year : IV} \\
\textbf{Tutorial \ : 1} \hfill \textbf{Part : II } \\
\textbf{Practical : 3/2}  \\

\par
\noindent 
\section*{Course Objectives:}
To understand the fundamentals of Radio Frequency (RF) and Microwave (M/W) theory and applications, design and analysis practices, and measurement techniques.


\mylist{
    \textbf{Introduction \hfill (3 hours)}
    \mylist{
        Standard Frequency bands;
        Behaviour of circuits at conventional and RF/Microwave bands;
        Microwave applications
    };
    \textbf{RF and M/W Transmission Lines \hfill (6 hours)}
    \mylist{
        Types of transmission lines;
        Transmission line theory;
        Smith Chart analysis;
        Impedance transformations and matching analysis
    };
    \textbf{RF and M/W Network Theory and Analysis \hfill (4 hours)}
    \mylist{
        Scattering matrix and its properties;
        S-parameter derivation and analysis
    };
    \textbf{RF/Microwave Components and Devices \hfill (8 hours)}
    \mylist{
        Coupling probes;
        Coupling loops;
        Waveguide;
        Termination, E-plane Tee, H-plane Tee, Magic Tee;
        Phase-Shifter;
        Attenuators;
        Directional coupler;
        Gunn diode;
        Microwave transistor;
        MASER;
        Resonator and circulators
    };
    \textbf{Microwave Generators \hfill (5 hours)}
    \mylist{
        Transit-time effect;
        Limitations of conventional tubes;
        Two-cavity and multi-cavity klystrons;
        Reflex klystron;
        TWT and magnetrons
    };
    \textbf{RF Design Practices \hfill (10 hours)}
    \mylist{
        RF Low pass filter
        \mylist{
            Insertion loss;
            Frequency scaling;
            Microstrip implementation
        };
        RF Amplifier
        \mylist{
            Amplifier theory;
            Design and real world consideration
        };
        Oscillator and mixer
        \mylist{
            Oscillator and super mixing theory;
            Design and real world consideration
        }
    };
    \textbf{Microwave Antennas and Propagation \hfill (3 hours)}
    \mylist{
        Antenna types;
        Propagation characteristics of microwave antennas;
        RF and M/W radiation, safety practices and standards
    };
    \textbf{RF/Microwave Measurements \hfill (6 hours)}
    \mylist{
        Power measurement;
        Calorimeter method;
        Bolometer bridge method;
        Thermocouples;
        Impedance measurement;
        RF frequency measurement and spectrum analysis;
        Measurement of unknown loads;
        Measurement of reflection coefficient;
        VSWR and Noise
    }
}


\section*{Practicals:}
\mylist{
    Illustration of Smith Chart and load analysis;
    Introduction to RF and M/W signal and circuits, measuring techniques, instrumentation, and practices;
    Designing and analysis of simple strip-line and two-port circuits using network and spectrum analysers;
    Software-based (ADS-like) RF signal \& circuit simulation practices
}

\section*{References:}
\mylist{
    Herbert J. Reich and et. al., Van Nostard Reinhold, ``Microwave Principles";
    K.C. Gupta, ``Microwave Electronics", Tata McGraw Hill;
    A.K. Gautam, ``Microwave Engineering", S.K. Kataria \& Sons;
    D.C. Agrawal, ``Microwave Techniques", Tata McGraw Hill;
    R. Chatterjee, ``Elements of Microwave Engineering", Tata McGraw Hill;
    Samuel Y. Liao, `` Microwave Devices \& Circuits", PHI;
    David M. Pozer, ``Microwave Engineering", John Wiley \& Sons.;
    Newington ``ARRL UHF/Microwave Experimenter's Manual", CT;
    W.H. Hayt, ``Engineering Electromagnetics", McGraw Hill Book Company;
    A. Das, ``Microwave Engineering", Tata McGraw Hill;
    William Sinnema, ``Electronic Transmission Technology: Lines, Waves, and Antennas", Prentice Hall.
}