\begin{center}
    \textbf{\huge{\uppercase{Computer Networks}}}
    \\
    \vspace{.5cm}
    \textbf{\large{CT 657}}
\end{center}

\noindent\textbf{Lecture\ \ \ : 3} \hfill \textbf{Year : III} \\
\textbf{Tutorial \ : 1} \hfill \textbf{Part : II } \\
\textbf{Practical : 3}  \\

\par
\noindent 
\section*{Course Objective:}
To understand the concepts of computer networking, functions of different layers and protocols, and know the idea of IPV6 and security.

\mylist{
    \textbf{Introduction to Computer Network \hfill (5 hours)}
    \mylist{
        Uses of Computer Network;
        Networking model Client/Server, p2p, active network;
        Protocols and Standards;
        OSI model and TCP/IP model;
        Comparison of OSI and TCP/IP model;
        Example network: The Internet, X.25, Frame relay, Ethernet, VoIP, NGN and MPLS, xDSL
    };
    \textbf{Physical Layer \hfill (5 hours)}
    \mylist{
        Network monitoring: delay, latency, throughput;
        Transmission media: Twisted pair, Coaxial, Fiber optic, Line-of-site, Satellite;
        Multiplexing, Circuit switching, Packet switching, VC Switching, Telecommunication switching system (Networking of Telephone exchanges);
        ISDN: Architecture, Interface, and Signaling
    };
    \textbf{Data Link Layer \hfill (5 hours)}
    \mylist{
        Functions of Data Link Layer;
        Framing;
        Error Detection and Corrections;
        Flow Control;
        Example of Data Link Protocol, HDLC, PPP;
        The Medium Access Sub-layer;
        The channel allocation problem;
        Multiple Access Protocols;
        Ethernet;
        Networks: FDDI, ALOHA, VLAN, CSMA/CD, IEEE 802.3, 802.4, 802.5 and 802.11
    };
    \textbf{Network Layer \hfill (9 hours)}
    \mylist{
        Inter-networking and devices: Repeaters, Hubs, Bridges, Switches, Router, Gateway;
        Addressing: Internet address, classful address;
        Subnetting;
        Routing: techniques, static vs dynamic routing, routing table in classful address;
        Routing protocols: RIP, OSPF, BGP, Unicast and multicast routing protocols;
        Routing algorithms: Shortest path algorithm, flooding, distance vector routing, link state routing{;}, Protocols: ARP, RARP, IP, ICMP
    };
    \textbf{Transport Layer \hfill (5 hours)}
    \mylist{
        The transport service: Services provided to the upper layers;
        Transport protocols: UDP, TCP;
        Port and Socket;
        Connection establishment, Connection release;
        Flow control and buffering;
        Multiplexing and de-multiplexing;
        Congestion control algorithm: Token Bucket and Leaky Bucket
    };
    \textbf{Application Layer \hfill (5 hours)}
    \mylist{
        Web: HTTP and HTTPS;
        File Transfer: FTP, PuTTY, WinSCP;
        Electronic Mail: SMTP, POP3, IMAP;
        DNS;
        P2P applications;
        Socket programming;
        Application server concept: proxy caching, Web/Mail/DNS server optimization;
        Concept of traffic analyzer: MRTG, PRTC, SNMP, Packet tracer, WireShark
    };
    \textbf{Introduction to IPV6 \hfill (4 hours)}
    \mylist{
        IPv6 Advantages;
        Packet formats;
        Extension headers;
        Transition from IPv4 to IPv6: Dual stack, Tunneling, Header Translation;
        Multicasting
    };
    \textbf{Network Security \hfill (7 hours)}
    \mylist{
        Properties of secure communication;
        Principles of cryptography: Symmetric Key and Public Key;
        RSA Algorithm;
        Digital Signatures;
        Securing e-mail (PGP);
        Securing TCP connection (SSL);
        Network Layer Security (IPsec, VPN);
        Securing wireless LANs (WEP);
        Firewalls: Application Gateway and Packet Filtering, and IDS
    }
}

\section*{Practical:}
\mylist{
    Network wiring and LAN setup;
    Router Basic Configuration;
    Static and Dynamic Routing;
    Creating VLAN;
    Router access-list configuration;
    Basic Network setup on Linux;
    Setup of Web Server;
    DNS Server setup;
    Setup of DHCP server;
    Virtualizations
}

\section*{References:}
\mylist{
    A.S. Tanenbaum, ``Computer Networks", Prentice Hall;
    W. Stallings, ``Data and Computer Communications", McMillian press;
    Kurose Ross, ``Computer Networking: A top down approach", Pearson Education;
    Larry L. Peterson, Bruce S. Davie, ``Computer Networks: A Systems Approach", Morgan Kaufmann Publishers.
}