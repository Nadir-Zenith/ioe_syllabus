\begin{center}
    \textbf{\huge{\uppercase{Filter Design}}}
    \\
    \vspace{.5cm}
    \textbf{\large{EX 704}}
\end{center}

\noindent\textbf{Lecture\ \ \ : 3} \hfill \textbf{Year : IV} \\
\textbf{Tutorial \ : 1} \hfill \textbf{Part : I } \\
\textbf{Practical : 3/2}  \\

\par
\noindent 
\section*{Course Objective:}
To familiarize student with the concept of analog filter design: passive filters, RC active filters and switched-capacitor filters

\mylist{
    \textbf{Introduction \hfill (4 hours)}
    \mylist{
        Filter and its importance in communication;
        Kinds of filters in terms of frequency response;
        Ideal response and response of practical filters;
        Normalization and de-normalization in filter design;
        Impedance (magnitude) scaling and frequency scaling;
        History of filter design and available filter technologies
    };
    \textbf{Approximation Methods \hfill (8 hours)}
    \mylist{
        Approximation and its importance in filter design;
        Lowpass approximations methods;
        Butterworth response, Butterworth pole locations, Butterworth filter design from specifications;
        Chebyshev and inverse Chebyshev characteristics, network functions and pole zero locations;
        Characteristics of Cauer (elliptic) response;
        Bessel-Thomson approximation of constant delay;
        Delay Equalization
    };
    \textbf{Properties and Synthesis of Passive Networks \hfill (7 hours)}
    \mylist{
        One-port passive circuits
        \mylist{
            Properties of passive circuits, positive real functions;
            Properties of lossless circuits;
            Synthesis of LC one-port circuits, Foster and Cauer circuits;
            Properties and synthesis of RC one-port circuits
        };
        Two-port passive circuits
        \mylist{
            Properties of passive two-port circuits, residue condition, transmission zeros;
            Synthesis of two-port LC and RC ladder circuits based on zero-shifting by partial pole removal
        }
    };
    \textbf{Design of Resistively-terminated lossless filter \hfill (4 hours)}
    \mylist{
        Properties of resistively-terminated lossless ladder circuits, transmission and reflection coefficients;
        Synthesis of LC ladder circuits to realize all-pole lowpass functions;
        Synthesis of LC ladder circuits to realize functions with finite transmission zeros
    };
    \textbf{Active Filter \hfill (7 hours)}
    \mylist{
        Fundamentals of Active Filter Circuits
        \mylist{
            Active filter and passive filter;
            Ideal and real operational amplifiers, gain-bandwidth product;
            Active building blocks: amplifiers, summers, integrator;
            First order passive sections and active sections using inverting and non-inverting op-amp configuration
        };
        Second order active sections (biquads)
        \mylist{
            Tow-Thomas biquad circuit, design of active filter using Tow-Thomas biquad;
            Sallen-Key biquad circuit and Multiple-feedback biquad (MFB) circuit;
            Gain reduction and gain enhancement;
            RC-CR transformation
        }
    };
    \textbf{Sensitivity \hfill (3 hours)}
    \mylist{
        Sensitivity and importance of sensitivity analysis;
        Definition of single parameter sensitivity;
        Centre frequency and Q-factor sensitivity;
        Sensitivity properties of biquads;
        Sensitivity of passive circuits
    };
    \textbf{Design of High-Order Active Filters \hfill (6 hours)}
    \mylist{
        Cascade of biquads
        \mylist{
            Sequencing of filter blocks, centre frequency, Q-factor and gain 
        };
        Active simulation of passive filters
        \mylist{
            Ladder design with simulated inductors;
            Ladder design with frequency-dependent negative resistors (FDNR);
            Leapfrog simulation of ladders
        }
    };
    \textbf{Switched-Capacitor Filters \hfill (4 hours)}
    \mylist{
        The MOS switch and switched capacitor;
        Simulation of resistor by switched capacitor;
        Switched-capacitor circuits for analog operations: addition, subtraction, multiplication and integration;
        First-order and second-order switched-capacitor circuits
    }
}


\section*{Practical:}
The laboratory experiments consist computer simulation as well hardware realization for analysis and design of passive and active filters which include.
\begin{itemize}
    \item Analysis and design of passive and active filter circuits using computer simulation
    \item Design of active filters using biquad circuits
    \item Design of higher order active filters using inductor simulation
    \item Design of higher order active filters using functional simulation
\end{itemize}


\section*{References:}
\mylist{
    Rolf Schaunmann, Mac E. Van Valkenburg, ``Design of Analog Filters";
    Wai-Kai Chen, `` Passive and Active Filters (Theory and Implementations)";
    Kendal L. Su., ``Analog Filter"
}