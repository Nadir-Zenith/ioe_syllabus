\begin{center}
    \textbf{\huge{\uppercase{Basic Electronics Engineering}}}
    \\
    \vspace{.5cm}
    \textbf{\large{EX 451}}
\end{center}

\noindent\textbf{Lecture\ \ \ : 3} \hfill \textbf{Year : I } \\
\textbf{Tutorial \ : 1} \hfill \textbf{Part : II } \\
\textbf{Practical : 3/2}  \\

\par
\noindent 
\section*{Course Objective:}
To understand the language of electronics, elements and their functionality. To understand analog and digital systems and their applications.

\begin{enumerate}
    \item \textbf{Basic Circuits Concepts \hfill (4 hours)}
    \begin{enumerate}
        \item Passive components: Resistance, Inductance, Capacitance; series, parallel combinations; Kirchhoff's law: voltage, current; linearity
        \item Signal sources: voltage and current sources; nonideal sources; representation under assumption of linearity; controlled sources: VCVS, CCVS, VCCS, CCCS; concept of gain, transconductance, transimpedance.
        \item Superposition theorem; Thevenin's theorem; Norton's theorem
        \item Introduction to filter
    \end{enumerate}
    
    \item \textbf{Diodes \hfill (6 hours)}
    \begin{enumerate}
        \item Semiconductor diode characteristics
        \item Modeling the semiconductor diode
        \item Diode circuits: clipper; clamper circuits
        \item Zener diode, LED, Photodiode, varacters diode, Tunnel diodes
        \item DC power supply: rectifier-half wave, full wave (center tapped, bridge), Zener regulated power supply
    \end{enumerate}
    
    \item \textbf{Transistor \hfill (8 hours)}
    \begin{enumerate}
        \item BJT configuration and biasing, small and large signal model
        \item T and $\mu$ model
        \item Concept of differential amplifier using BJT
        \item BJT switch and logic circuits
        \item Constructions and working principle of MOSFET and CMOS
        \item MOSFET as logic circuits
    \end{enumerate}
    
    \item \textbf{The Operational Amplifier and Oscillator \hfill (7 hours)}
    \begin{enumerate}
        \item Basic model; virtual ground concept; inverting amplifier; non-inverting amplifier; integrator; differentiator, summing amplifier and their applications 
        \item Basic feedback theory; positive and negative feedback; concept of stability; oscillator
        \item Waveform generator using op-amp for Square wave, Triangular wave Wien Bridge oscillator for sinusoidal waveform
        
    \end{enumerate}
    
    \item \textbf{Communication System \hfill (4 hours)}
    \begin{enumerate}
        \item Introduction 
        \item Wired and wireless communication system
        \item EMW and propagation, antenna, broadcasting and communication 
        \item Internet/intranet
        \item Optical fiber
    \end{enumerate}
    
    \item \textbf{Digital Electronics \hfill (11 hours)}
    \begin{enumerate}
        \item Number systems, Binary arithmetic
        \item Logic gates: OR, NOT, AND, NOR, NAND, XOR, XNOR gate; Truth tables
        \item Multiplexers; Demux, Encoder, Decoder
        \item Logic function representation
        \item Combinational Circuits: SOP, POS form; K-map;
        \item Latch, flip-flop; S-R flip-flop; JK master slave flip-flop; D-flip flop
        \item Sequential circuits: Generic block diagram; shift registers; counters
    \end{enumerate}
    
    \item \textbf{Application of Electronics System \hfill (5 hours)}
    \begin{enumerate}
        \item Instrumentation system: Transducer, strain guage, DMM, Oscilloscope
        \item Regulated power supply
        \item Remote control, character display, clock, counter, measurements, date logging, audio video system
    \end{enumerate}
\end{enumerate}

\section*{Practical}
\begin{enumerate}
    \item Familiarization with passive components, function generator and oscilloscope
    \item Diode characteristics, rectifiers, Zener diodes
    \item Bipolar junction transistor characteristics and single stage amplifier
    \item Voltage amplifiers using op-amp, Comparators, Schmitt
    \item Wave generators using op-amp
    \item Combinational and sequential circuits
\end{enumerate}


\section*{References:}
\begin{enumerate}
    \item Robert Boylestad and Lous Nashelsky, ``Electronics Devices and Circuit Theory", PHI
    \item Thomas L. Floyd, ``Electronic Devices", Pearson Education, Inc. 2007
    \item A.S. Sedra and K.C. Smith, ``Microelectronics Circuits", Oxford University Press, 2006
\end{enumerate}