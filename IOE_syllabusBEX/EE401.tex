\begin{center}
    \textbf{\huge{\uppercase{Basic Electrical Engineering}}}
    \\
    \vspace{.5cm}
    \textbf{\large{EE 401}}
\end{center}

\noindent\textbf{Lecture\ \ \ : 3} \hfill \textbf{Year : I } \\
\textbf{Tutorial \ : 1} \hfill \textbf{Part : I } \\
\textbf{Practical : 3/2}  \\

\par
\noindent 
\section*{Course Objective:}
To understand the fundamental concept of DC, AC \& 3-phase electrical circuits.

\begin{enumerate}
    \item \textbf{ General Electric System \hfill (6 hours)}
    \begin{enumerate}
        \item Constituent parts of an electrical system (source, load, communication \& control)
        \item Current flow in a circuit
        \item Electromotive force and potential difference
        \item Electrical units
        \item Ohm's law
        \item Resistors, resistivity
        \item Temperature rise \& temperature coefficient of resistance
        \item Voltage \& current sources
    \end{enumerate}
    
    \item \textbf{ DC circuits \hfill (4 hours)}
    \begin{enumerate}
        \item Series circuits
        \item Parallel networks
        \item Krichhoff's laws
        \item Power and energy
    \end{enumerate}
    
    \item \textbf{ Network Theorems \hfill (12 hours)}
    \begin{enumerate}
        \item Application of Krichhoff's laws in network solution
        \begin{enumerate}
            \item Nodal Analysis
            \item Mesh Analysis
        \end{enumerate}
        \item Star-delta \& delta-star transformation
        \item Superposition theorem
        \item Thevninn's theorem
        \item Nortan's theorem
        \item Maximum power transfer theorem
        \item Reciprocity theorem
    \end{enumerate}
    
    \item \textbf{ Inductance \& Capacitance in electric circuits \hfill (4 hours)}
    \begin{enumerate}
        \item General concept of capacitance
        \begin{enumerate}
            \item Charge \& voltage
            \item Capacitors in series and parallel
        \end{enumerate}
        
        \item General concept of inductance
        \begin{enumerate}
            \item Inductive \& non-inductive circuits
            \item Inductance in series \& parallel
        \end{enumerate}
    \end{enumerate}
    
    \item \textbf{ Alternating Quantities \hfill (2 hours)}
    \begin{enumerate}
        \item AC systems
        \item Wave form, terms \& definitions
        \item Average and rms values of current \& voltage
        \item Phasor representation
    \end{enumerate}
    
    \item \textbf{ Single-phase AC Circuits \hfill (6 hours)}
    \begin{enumerate}
        \item AC in resistive circuits
        \item Current \& voltage in an inductive circuits
        \item Current and voltage in an capacitive circuits
        \item Concept of complex impedance and admittance
        \item AC series and parallel circuit
        \item RL, RC and RLC circuit analysis \& phasor representation
    \end{enumerate}
    
    \item \textbf{ Power in AC Circuits \hfill (5 hours)}
    \begin{enumerate}
        \item Power in resistive circuits
        \item Power in inductive and capacitive circuits
        \item Power in circuit with resistance and reactance
        \item Active and reactive power
        \item Power factor, its practical importance
        \item Improvement of power factor
        \item Measurement of power in a single-phase AC circuits
    \end{enumerate}
    
    \item \textbf{Three-Phase Circuit Analysis \hfill (6 hours)}
    \begin{enumerate}
        \item Basic concept \& advantage of Three-phase circuit
        \item Phasor representation of star and delta connection
        \item Phase and line quantities
        \item Voltage \& current computation in 3-phase balance and unbalance circuits
        \item Real and reactive power computation
        \item Measurement of power and power factor in 3-phase system
    \end{enumerate}
    
\end{enumerate}

\section*{Practical:}
\begin{enumerate}
    \item Measurement of Voltage, Current and power in DC circuit, \\
    Verification of Ohm's law \\ Temperature effects in Resistance
    
    \item Krichhoff's Voltage and Current law \\ Evaluate power from V and I \\ Note loading effects of meter
    
    \item Measurement amplitude, frequency and time with oscilloscope \\
    Calculate and verify average and rms value \\ 
    Examine phase relation in RL \& RC circuit
    
    \item Measurements of alternating quantities \\ R, RL, RC circuits with AC excitation \\
    AC power, power factor, VARs, phasor diagrams
    
    \item Three-phase AC circuits \\ Measure currents and voltages in three-phase balanced AC circuits \\ Prove Y-$\Delta$ transformation \\ Exercise on phasor diagrams for three-phase circuits 
    
    \item Measurement of Voltage, current and power in a three-phase circuit \\ Two-wattmeter method of power measurement in R, RL and RC three phase circuits \\ Watts ratio curve 
\end{enumerate}


\section*{References:}
\begin{enumerate}
    \item J.R Cogdell, ``Foundations of Electrical Engineering", Prentice Hall, Englewood Chiffs, New Jersy
    
    \item I.M. Smith, ``Haughes Electrical Technology", Addison-Wesley, ISR Rprint
\end{enumerate}