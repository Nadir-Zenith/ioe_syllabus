\begin{center}
    \textbf{\huge{\uppercase{Propagation and Antenna}}}
    \\
    \vspace{.5cm}
    \textbf{\large{EX 653}}
\end{center}

\noindent\textbf{Lecture\ \ \ : 3} \hfill \textbf{Year : III} \\
\textbf{Tutorial \ : 1} \hfill \textbf{Part : II } \\
\textbf{Practical : 3/2}  \\

\par
\noindent 
\section*{Course Objective:}
To provide the student with an understanding of antennas, EM wave propagation and optical fiber communications.

\mylist{
    \textbf{Radiation and Antenna Fundamentals \hfill (5 hours)}
    \mylist{
        Retarded Potentials: EM wave generation with a conduction current, the short uniform current dipole, the radiate electric and magnetic fields.;
        Radiation patterns and input impedance of the short uniform current dipole, the short Dipole and long dipole;
        Antenna theorems: reciprocity, superposition, Thevenin, minimum power transfer, compensation, equality of directional patterns, equivalence of receiving and transmitting impedances.
    };
    \textbf{Antenna Parameters and Arrays \hfill (5 hours)}
    \mylist{
        Basic antenna parameters:
        Pattern multiplication: Linear and two-dimensional antenna arrays, end fire and Broadside arrays.
    };
    \textbf{Antennas classification \hfill (10 hours)}
    \mylist{
        Isotropic antenna;
        Omni directional antenna: Dipole;
        Directional antennas;
        Travelling wave antennas - single wire, V and Rhombus Reflector antennas -- large plane sheet, small plane sheet, linear, corner, parabolic, elliptical, hyperbolic and circular reflector. Aperture antenna -- horn array antennas -- Yagi-Uda, Lag periodic, other antennas -- Monopole, Loop, Helical, Microstrip.
    };
    \textbf{Propagation and Radio Frequency Spectrum \hfill (7 hours)}
    \mylist{
        Ground or surface wave;
        Space wave: direct and ground reflected wave, duct propagation;
        Ionospheric or sky wave: critical frequency, MUF, Skip distance;
        Tropospheric wave;
        Radio frequency spectrum and its propagation characteristics
    };
    \textbf{Propagation between Antennas: \hfill (7 hours)}
    \mylist{
        Free space propagation: power density of the receiving antenna, path loss;
        Plane earth propagation: the ground reflection, effective antenna heights, the two ray;
        Propagation model, path loss;
        Fresnel Zones and Knife edge diffraction
    };
    \textbf{Optical Fibers (Introductory) \hfill (11 hours)}
    \mylist{
        Optical fiber communication system and its advantages and disadvantages over metal wire communication system;
        Types of optical fibre and its structural difference;
        Light propagation characteristics and Numerical Aperture (NA) in optical fiber;
        Losses;
        Light source and photo detector
    }
}



\section*{Practical:}
\mylist{
    Two experiments in properties of EM waves: refraction, diffraction, polarization;
    Two experiments in radiation patterns of various types of antennas;
    Two experiments in measurements on optical fiber transmission systems
}

\section*{References:}
\mylist{
    J.D. Kraus, ``Antenna", McGraw Hill;
    C.A. Balanis, ``Antenna Theory Analysis and Design", John Wiley and Sons Inc.;
    Collins, R.E., ``Antenna and Radio Wave Propagation", McGraw Hill;
    Gerd Kaiser, ``Optical Fiber Communications", McGraw Hill;
    John Gowar, ``Optical Communication Systems", PHI Publications
}